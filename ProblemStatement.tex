\documentclass[12pt]{article}
\usepackage{nopageno}
\usepackage{xcolor} % for different colour comments
\usepackage{parskip} % Space between each paragraph.
%\usepackage{hardwrap} % for text length of 80 pts
\usepackage[margin=1.2in]{geometry}
\usepackage{ltx/edcomms}

\setlength{\parindent}{15pt} % parskip sets this to 0. 15 is default.

%% Comments
\newif\ifcomments\commentsfalse

%% Dr. Kahl's comment package. Uncomment the following line to hide comments.
\edcommsfalse

\ifcomments
\newcommand{\authornote}[3]{\textcolor{#1}{[#3 ---#2]}}
\newcommand{\todo}[1]{\textcolor{red}{[TODO: #1]}}
\else
\newcommand{\authornote}[3]{}
\newcommand{\todo}[1]{}
\fi

% wss = Dr. Smith ; ds = Dr. Szymczak
\newcommand{\wss}[1]{\authornote{magenta}{SS}{#1}}
\newcommand{\ds}[1]{\authornote{blue}{ND}{#1}}


%%%%%%%%%%%%%%%	START OF DOCUMENT %%%%%%%%%%%%%%%%%%%%

\begin{document}
\title{CS4ZP6 Problem Statement \\ Ampersand Tarski Event-Condition-Action Rules } 
\author{Yuriy Toporovskyy,\ Yash Sapra,\ Jaeden Guo}
\date{September 25, 2015}
\thispagestyle{empty}
\maketitle
\wss{This is an example comment.  You can turn comments off by replacing
		commentstrue by commentsfalse.}
\edcomm{YT}{
What is EFA? What is ECA? These should be defined before they are used;
so, they shouldn't appear in the title. In fact we should only mention it in
passing; overuse of acronyms can make a text undecipherable. I know what ECA
is, but I don't know what EFA is and it is never defined in this document, at 
all.}
\edcomm{JG}{ EFA turned out to be the acronym for the project which was the 
title E.F.A (Event-Condition Action) rules for Ampersand on the Project 
Registration. It was on there you signed it and when I asked you two for a 
project title, neither of you said anything and it was due }
\edcomm{YS}{I thought its was ECA, my bad. I don't think we can change what's done}

\edcomm{Note}{You don't need to put /indent before every paragraph. Latex inserts
that automatically. You can change the amount by changing the value of
/parindent.  Also instead of manually adding a line break after every
paragraph, just use parskip (see above).}

The Ampersand Project translates the non-functional requirements of a
business into the functional specifications required by software engineers. It
also provides engineers with a variety of aids which help them to design
products that fulfill all of the needs of their clients and the end-users.
\edcomm{YT}{What does ``such as data models and service catalogues'' mean? If I
  don't understand it it, the prof won't}
\edcomm(JG){ from what I can tell ADL generates data models and service 
catalogues according to functional specifications, and ADL is part of 
Ampersand; I got that from Joosten's article -- we could just take it out.}
Business requirements are not universal and often vary from one country or
culture to the next. Additional difficulty arises from the fact that business
requirements are not easily translatable into design requirements.

The Ampersand Project has proven reliable at translating natural
language into technical specifications so that they could be incorporated into
the overall design. Business transactions input by the user are translated into
``process rules''; and requirements are translated into ``Event-Condition
Action'' (ECA) rules. Both of these rules are combined into a single SQL
program. Currently, a programmer using Ampersand must manually ensure that each
requirement is satisfied after each transaction, by telling Ampersand how each
violation must be fixed. Not only is this process error prone, it is
unmaintainable; when new requirements are added, the already-existing system may
violate the new restrictions. In many scenarios, the requirements may be safety
and security critical. To this end, Ampersand must also generate relational
algebra proofs to show that the SQL generated from ECA rules is sound. 

This project will focus primarily on modifying Ampersand to include the
translation from requirements to ECA rules to SQL commands (whereas the pipeline
for process rules is already in place). This will ameliorate Ampersand in two
ways; by automating part of the programmers job, namely writing code to preserve
requirements throughout a transaction; and by generating SQL code which is
proven to not violate any system requirements. 

\end{document}










