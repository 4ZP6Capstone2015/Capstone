\documentclass[12pt]{article}
\usepackage{nopageno}
\usepackage{xcolor} % for different colour comments
%\usepackage{hardwrap} % for text length of 80 pts

%% Comments
\newif\ifcomments\commentsfalse

\ifcomments
\newcommand{\authornote}[3]{\textcolor{#1}{[#3 ---#2]}}
\newcommand{\todo}[1]{\textcolor{red}{[TODO: #1]}}
\else
\newcommand{\authornote}[3]{}
\newcommand{\todo}[1]{}
\fi

% wss = Dr. Smith ; ds = Dr. Szymczak
\newcommand{\wss}[1]{\authornote{magenta}{SS}{#1}}
\newcommand{\ds}[1]{\authornote{blue}{ND}{#1}}
\newcommand{\jg}[1]{\authornote{green}{JG}{#1}}


\wss{This is an example comment.  You can turn comments off by replacing
		commentstrue by commentsfalse.}
%%%%%%%%%%%%%%%	START OF DOCUMENT %%%%%%%%%%%%%%%%%%%%

\begin{document}
\title{Problem Statement for E.F.A. (E.C.A for Ampersand)} 
\author{Yuriy Toporovskyy,\ Yash Sapra,\ Jaeden Guo}
\date{September 24, 2015}
\thispagestyle{empty}
\maketitle

\indent The Ampersand Project translates the non-functional requirements of a 
business into the functional specifications required by engineers. It provides 
engineers with a variety of aids, such as data models and service catalogues, 
which help them to design products that fulfill all of the needs of their 
clients and the end-users. Business requirements are not universal and often 
vary from one country or culture to the next. The difficulty is that business 
requirements are not easily translatable into design requirements.\\
\indent The Ampersand Project is important because it has proven reliable at 
translating natural language into technical specifications so that they could 
be incorporated into the overall design. E.F.A. will translate user 
instructions into ECA (Event-Condition Action) rules so that Ampersand can 
restructure existing data. It is crucial that these rules be absolute; although 
the user will be able to alter the rules, it must not be possible for the data 
within the system to bypass or break these rules. \\
\indent The E.F.A. project will boost the efficiency of Ampersand by reducing 
the need for manual adjustments by restructuring data in accordance with the 
new rules. The E.F.A project will create an extension to Ampersand that 
restores the validity of data in accordance with any alterations made by the 
user. \\ 
\indent Currently, there is no system to check the correctness of the data 
once new rules are added, and thus previously-existing system data may violate the 
new restrictions. E.F.A will compensate for these violations by automating the 
restructuring of data sets in accordance with the new regulations. Thus the 
successfully completed project will act as a correctness test for the Ampersand 
system, in addition to boosting its efficiency. 




\end{document}










