\documentclass[12pt]{report}
\usepackage{xcolor} % for different colour comments
\usepackage{parskip} % Space between each paragraph.
%\usepackage{hardwrap} % for text length of 80 pts
\usepackage[margin=1.2in]{geometry}
\usepackage{hyperref}
\usepackage{../ltx/edcomms}
\usepackage{graphicx}
\usepackage[section]{placeins} % Prevents floats from floating across sections
\usepackage{natbib}%Bibtex
\usepackage{float}
\usepackage{tabularx}
\usepackage{ltablex} %% Multi page tables 
\usepackage{booktabs}
\usepackage{tabto}
\usepackage{tocloft} %% This package prevents table of contents from generating a page break
\usepackage{caption}
\usepackage{ifthen}

%% Comments are enabled and disabled by 'draft' mode. I hacked in my own draft
%% mode (https://en.wikibooks.org/wiki/LaTeX/Macros) because the LaTeX draft
%% mode disables a bunch of things that I don't want it to. I just want it to
%% disable comments. Do not set any of this manually, just use the build script,
%% which builds both draft and final copies. Comments are enabled by default, so
%% if you build manually, you get a draft copy. 
\providecommand\draftmode{true}

\ifthenelse{\equal{\draftmode}{true}}{
\newcommand{\authornote}[3]{\textcolor{#1}{[#3 ---#2]}}
\newcommand{\todo}[1]{\textcolor{red}{[TODO: #1]}}
\edcommstrue %% Dr. Kahl's comment package. Eventually we should migrate all
             %% comments to this.
}{
\newcommand{\authornote}[3]{}
\newcommand{\todo}[1]{}
\edcommsfalse 
}

% wss = Dr. Smith ; ds = Dr. Szymczak
\newcommand{\wss}[1]{\authornote{magenta}{SS}{#1}}
\newcommand{\ds}[1]{\authornote{blue}{DS}{#1}}


\usepackage{geometry}
\usepackage{changepage}
\usepackage{adjustbox}
\setlength{\parindent}{15pt} % parskip sets this to 0. 15 is default.

\newcolumntype{C}[1]{>{\centering}p{#1}} %% For use with tabularx
%%%%%%%%%%%%%%%	START OF DOCUMENT %%%%%%%%%%%%%%%%%%%%
\edcommsfalse
\begin{document}

\pagenumbering{roman} %% Roman numerals before actual document starts
\begin{titlepage}\begin{center}
\thispagestyle{empty} %% No page no. on title

\vspace*{1cm}

{\Huge\textbf{Ampersand Event-Condition-Action Rules}}

\vspace{0.5cm}
{\Large Design Document
	
	\edinsert{JG}{Version 0}

\vspace{1.5cm}
Yuriy Toporovskyy (toporoy) \\ Yash Sapra (sapray) \\ Jaeden Guo (guoy34)}
\vfill


\vspace{0.8cm}
\end{center}
CS 4ZP6 \\
February 29th, 2016 \\ 
Fall 2015 / Winter 2016 
\end{titlepage}

%% Revision history

\begin{table}[ht!]\begin{center}
\caption{Revision History}  
\begin{tabular}{|l|l|l|}\hline
\textbf{Author} & \textbf{Date} & \textbf{Comment} \\\hline 
Yash Sapra & 24 / 02 / 2016 & Initial draft\\\hline
\end{tabular}
\end{center}\end{table}

\newpage

\tableofcontents
\listoffigures
\listoftables

\newpage
\pagenumbering{arabic} %% Arabic numerals in actual document

%%%%%%%%%%%%%%%%%%%%%%%%%%%%%%%%%%%%
%% Chapter 1: Introduction %%
%%%%%%%%%%%%%%%%%%%%%%%%%%%%%%%%%%%%
\chapter{Introduction}\label{ch:Introduction}

\section{Purpose}\label{sec:Purpose}
The document outlines the design decision for the EFA project. 
EFA is responsible for generating SQL from ECA rules that will 
be used to fixed any data inconsistencies in the Ampersand Database.

%% TODO :  Build Chapters as we go
%\section{Design Principles}\label{sec:DesignPrinciples}
%\section{Document Structure}\label{sec:DocumentStructure}

\chapter{Modules - Temporary }\label{ch:Modules_Temp}
%% Keeping these tables here for a bit, will have to put them in Appropiate section
%%\subsection{Partner or Collaborative Applications}\label{subsec:Collaborative}
This section provides a brief description of the modules that EFA uses. More 
details are provided in the Module Guide for EFA.

\begin{adjustbox}{center}
\begin{tabular}{ |p{3.2cm}|p{11cm}|  }
    \hline
    \multicolumn{2}{|c|}{\bfseries{\large{Ampersand Core Modules}}} \\ 
    \hline\hline
    \bfseries{Module Name} & \bfseries{Description}\\
    \hline
    AbstractSyntaxTree   & Ampersand's abstract representation of input from 
    ADL.   \\
    \hline
    FSpec &   A module that represents the F-Spec structure.\\
    \hline
    Basics & An Ampersand internal module that provides basic functions for 
    data manipulation.\\
    \hline
    ParseTree    & Ampersand parse tree for ADL script; concrete representation 
    of the input, and retains all information of the input. \\
    \hline
\end{tabular}
\end{adjustbox}


\begin{adjustbox}{center}
\begin{tabular}{ |p{3.2cm}|p{11cm}|  }
    \hline
    \multicolumn{2}{|c|}{\bfseries{\large{External Haskell Modules}}} \\
    \hline\hline
    \bfseries{Module Name} & \bfseries{Description}\\
    \hline
    GHC.TypeLits   & Internal GHC module used in the implementation of 
    type-level natural numbers \cite{hackage}    \\ 
    \hline
    Data.List & A module that provides support for operations on list 
    structures.  \\
    \hline
    Data.Char &  A module that provides support for characters and operations 
    on characters.  \\
    \hline
    Data.Coerce  & Provides safe coercions between data types; allows user to 
    safely convert between values of type that have the same representation 
    with no run-time overhead \cite{hackage}.\\
    \hline
    Debug.Trace &  Interface for tracing and monitoring execution, used for 
    investigating bugs and other performance issues \cite{hackage}. \\ 
    \hline 
    GHC.TypeLits & Internal GHC module that declares the constants used in 
    type-level implementation of natural numbers \cite{hackage}.  \\ 
    \hline  
    SimpleSQL.Syntax&   The AST for SQL queries \cite{hackage}\\ 
    \hline
    Text.PrettyPrint .Leijen & A pretty printer module based 
    off of Philip Wadler's 1997 "A prettier printer", used to show SQL queries 
    in a readable manner to humans \cite{hackage}.\\ 
    \hline
    GHC.Exts & This module provides access to types, classes, and functions 
    necessary to use GHC extensions. \\ 
    \hline
    Unsafe.Coerce &  A helper module that converts a value from any type to any 
    other type.This is used in the translation of ECA rules to SQL using 
    user-defined data types. \\ 
    \hline
     System.IO.Unsafe & IO computation must be free of side effects and 
     independent of its environment to be considered safe. Any I/O computation 
     that is wrapped in unsafePerformIO performs side effects. \\
    \hline
\end{tabular}
\end{adjustbox}

\begin{adjustbox}{center}
    \begin{tabular}{ |p{3.2cm}|p{11cm}|  }
        \hline
        \multicolumn{2}{|c|}{\bfseries{\large{EFA Modules}}} \\
        \hline\hline
        \bfseries{Module Name} & \bfseries{Description}\\
        \hline
        ECA2SQL & The top-level module that takes ECA rules from FSpec and 
        converts it into SQL queries.    \\ 
        \hline
        Equality & Various utilities related to type level equality   \\ 
        \hline
        PrettyPrinterSQL & Prints SQl queries in human-readable format    \\ 
        \hline
        Singletons & Module for Singleton datatypes, the module defines 
        singleton for a kind in terms of an isomorphism between a type and a 
        type representation   \\ 
        \hline
        Trace & Provides trace messages and various utilities used by 
        ECA2SQL    \\ 
        \hline
        TSQLCombinators & Uses an overloaded operator for indexing and 
        implements SQL as a primitive data type.   \\ 
        \hline
        TypedSQL & Contains basic SQL types represented in Haskell.    \\ 
        \hline
        Utils & Provides utility functions for ECA2SQL and contains type 
        families    \\ 
        \hline
    \end{tabular}
\end{adjustbox}


%\bibliography{SRS}
\end{document}










