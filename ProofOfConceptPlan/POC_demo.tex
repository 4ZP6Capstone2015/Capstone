\documentclass[12pt]{article}
\usepackage{nopageno}
\usepackage{xcolor} % for different colour comments
\usepackage{parskip} % Space between each paragraph.
\usepackage{hardwrap} % for text length of 80 pts
\usepackage[margin=1.2in]{geometry}

%\usepackage{../ltx/edcomms}
\usepackage{titling}

%\usepackage{ifthen}

%% Comments are enabled and disabled by 'draft' mode. I hacked in my own draft
%% mode (https://en.wikibooks.org/wiki/LaTeX/Macros) because the LaTeX draft
%% mode disables a bunch of things that I don't want it to. I just want it to
%% disable comments. Do not set any of this manually, just use the build script,
%% which builds both draft and final copies. Comments are enabled by default, so
%% if you build manually, you get a draft copy. 
\providecommand\draftmode{true}

\ifthenelse{\equal{\draftmode}{true}}{
\newcommand{\authornote}[3]{\textcolor{#1}{[#3 ---#2]}}
\newcommand{\todo}[1]{\textcolor{red}{[TODO: #1]}}
\edcommstrue %% Dr. Kahl's comment package. Eventually we should migrate all
             %% comments to this.
}{
\newcommand{\authornote}[3]{}
\newcommand{\todo}[1]{}
\edcommsfalse 
}

% wss = Dr. Smith ; ds = Dr. Szymczak
\newcommand{\wss}[1]{\authornote{magenta}{SS}{#1}}
\newcommand{\ds}[1]{\authornote{blue}{DS}{#1}}



\setlength{\parindent}{13pt} % parskip sets this to 0. 15 is default.
\setlength{\droptitle}{-9em}

\begin{document}

\title{CS4ZP6 Proof of Concept demo Plan\\ \vspace{2ex}} 
\author{\normalsize{Yuriy Toporovskyy,\ Yash Sapra,\ Jaeden Guo}}
\date{\normalsize\today \vspace{1ex}}
\thispagestyle{empty}
\maketitle
\paragraph{Ampersand Introduction}
Ampersand is a software and an approach for the use of business rules to define 
business processes. Though business rules aim to enhance the quality and 
efficiency of requirement elicitation in software projects there is still a 
huge gap between business rules are they are presented and the way they are 
implemented in software systems, and that is what Ampersand hopes to bridge. 
The main advantage being that Ampersand produces provably correct requirements' 
consistency and traceability.

Ampersand begins by describing business rules in a formal language 
\big(Abstract Data Language\big), and turns that into a functional 
specification, documentation and a working software prototype.   

\paragraph{Ampersand As it is}
Ampersand contains high-level architecture and is theoretically based off of 
mathematical concepts such as relational algebra and Tarski's axioms to name a 
few. The Ampersand process begins by taking the end-user writing business rules 
in a domain specific language \big(ADL\big), it is then parsed into a parse 
tree \big(referred to as the P-structure\big) which feeds into the type checker 
and converts it into relational algebra format \big(referred to as the 
A-structure\big). The semantics of ADL is expressed in terms of the A-structure 
then passed to the Calc component which generates the functional structure 
\big( referred to as the F-structure or F-spec\big). The F spec contains all 
necessary specification and generate the output, and the inner working of 
F-spec is the focus of our Capstone Project. 

\paragraph{Current Advancements made to Ampersand Software}
As mentioned previously one of the artifacts that Ampersand generates is a 
Database with all tables. ECA rules \big(or Event-Condition-Action rules\big) 
are used to maintain consistency of data. It is our task to translate these ECA 
rules to SQL statements which can be used to resolve any violation or 
inconsistency within the Ampersand System. In the previous system the "EXEC 
ENGINE" was a bandaid generally used for this purpose, but was very limited and 
missing many essential functionalities such as the ability to update. 

%I need someone to fill in the blanks here of what a delta is what the 
%refernece is --- reference for individual ECA rules?
The ECA rules are composed of a reference, a delta, a data type referred to as 
a PAClause. The PAClause indicates what is to be done and the delta can be 
thought of as a parameter for the ECArule that indicates which action needs to 
be taken. 

The PAClauses contain various actions that can be executed, and each is based 
on a guard that may trigger none, a few or all possible actions. These 
PAClauses maintain system invariants which must be consistent, we are bridging 
the SQL commands to bridge the actions to the meaning.

In order to capture the semantics and syntax of SQL, unique data types and 
methods had to be constructed to capture relationships between entities. Each 
component was deconstructed to fit the types represented in Haskell, include 
associated SQL types with their associated domain of interpretation. 

\paragraph{Yuriy -- freestyle, I can't explain future steps}
\begin{itemize}
	\item mockdata base
	\item magic?
	\item undefined evaluations?
\end{itemize}

\paragraph{original}
The focus of the Proof Of Concept will be to demonstrate the current 
advancements made to the Ampersand software that allow us to translate the 
Event - Condition - Action (ECA) rules to SQL statements which can be used be 
Ampersand to resolve any violations or inconsistencies in the Ampersand 
Database.
\newline\newline
\indent We will first give a brief introduction of the Ampersand software to 
the team attending our Proof of concept demo. This will allow team 2 to 
familiarize themselves with Ampersand and what it is capable of producing 
\big(i.e. 
Information Systems\big). We will first compile a simple ADL script in 
Ampersand to 
generate a working information system. At this point, once we have a working 
information system generated from Ampersand we we'll inform the audience about 
what other artifacts can be generated from the Ampersand software.

Once everyone in the audience is on board with Ampersand and its capabilities, 
we will go ahead and demonstrate our project. For this purpose, we'll use the 
same example that was used in the previous demonstration, which is a Project 
Administration Information System. The aim is to demonstrate the SQL that is 
generated from the ECA rules that Ampersand already generates. A manual 
translation of ECA to SQL will demonstrate the minimal correctness of the 
generated SQL. Once that is clear, the we'll inform the audience about what 
challenges lie ahead and what's the next milestone we will be working on.
\end{document}










