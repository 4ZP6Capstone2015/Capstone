\documentclass[12pt]{article}
\usepackage{nopageno}
\usepackage{xcolor} % for different colour comments
\usepackage{parskip} % Space between each paragraph.
\usepackage{hardwrap} % for text length of 80 pts
\usepackage[margin=1.2in]{geometry}
\usepackage{../ltx/edcomms}

\usepackage{ifthen}
\usepackage{../ltx/edcomms}

%% Comments are enabled and disabled by 'draft' mode. I hacked in my own draft
%% mode (https://en.wikibooks.org/wiki/LaTeX/Macros) because the LaTeX draft
%% mode disables a bunch of things that I don't want it to. I just want it to
%% disable comments. Do not set any of this manually, just use the build script,
%% which builds both draft and final copies. Comments are enabled by default, so
%% if you build manually, you get a draft copy. 
\providecommand\draftmode{true}

\ifthenelse{\equal{\draftmode}{true}}{
\newcommand{\authornote}[3]{\textcolor{#1}{[#3 ---#2]}}
\newcommand{\todo}[1]{\textcolor{red}{[TODO: #1]}}
%\edcommstrue %% Dr. Kahl's comment package. Eventually we should migrate all
             %% comments to this.
}{
\edcommsfalse 
\newcommand{\authornote}[3]{}
\newcommand{\todo}[1]{}
}

% wss = Dr. Smith ; ds = Dr. Szymczak
\newcommand{\wss}[1]{\authornote{magenta}{SS}{#1}}
\newcommand{\ds}[1]{\authornote{blue}{DS}{#1}}



\setlength{\parindent}{15pt} % parskip sets this to 0. 15 is default.


\begin{document}
\title{CS4ZP6 Test Plan \\ Ampersand Tarski Event-Condition-Action 
Rules \\ \vspace{-1ex}} 
\author{\normalsize{Yuriy Toporovskyy,\ Yash Sapra,\ Jaeden Guo}}
\date{\normalsize\today \vspace{-3ex}}
\thispagestyle{empty}
\maketitle
\paragraph{}
The focus of the test plan will be to demonstrate the current state of the 
Ampersand model prior to the addition of our project and to provide an overview 
of where our project fits into the Ampersand system. Ampersand is a system with 
patches for its unfinished pieces and one of those missing pieces is the 
purpose of our project, EFA. Although Ampersand is functional, it is limited in 
the number of finished products and complete services it can provide to its 
users. An example script will be compiled using Ampersand and multiple switches 
(i.e. compilation options) in order to generate different user feedback and 
create various Ampersand artifacts. 
\newline
\indent This demonstration will be a black box test of the current Ampersand 
system, one designed to produce various kinds of errors and violations. The 
error messages inform the user of syntax and definition errors, such as a 
missing variable. Violations inform the user when the model they are attempting 
to build violates pre-set conditions within the Ampersand system. An example of 
this is would be if a theatre is hiring actors to play a role, then a rule 
could be put in place stating that only actors who have a sufficient amount of 
experience can play lead. The user can define a ‘sufficient amount of 
experience’ to be anything they wish, from the number of plays in which the 
individual has acted to the number of years that the actor has worked in the 
industry. A violation of the rule would be an attempt to cast an actor in a 
leading role who has not met the conditions of a ‘sufficient amount of 
experience’. 
\newline
\indent The purpose of the EFA project is to automate the correction of a 
particular class of violations in order to restore invariants within the 
Ampersand system. These violations come from the misuse of PAClauses (Process 
Algebra Clauses); PAClauses are datatypes formed from ECA (Event-Condition 
Action) rules that are used to represent the structure of an active database 
system. In Ampersand, these ECA rules are deconstructed to their most basic 
forms which we call “Atoms”. Atoms are the simplest possible form a rule can 
take before losing its meaning. From here, a Haskell program (i.e., EFA) will 
automate the correction of violations by detecting contradictions between Atoms 
imposed on various datasets. For example, one Atom states that all entities in 
its set must be blue, another Atom states that all entities in its set must be 
red. The intersection between these two Atoms is red AND blue; if the user 
inserts something in the intersection of red and blue that is not red and blue, 
it is a violation of both Atoms. A quick fix to restore both rules would be to 
remove the entity that is not red or blue from the intersection of the red and 
blue set. 
\newline
\indent EFA is in its initial stages of development, and at this time it is 
difficult to provide a rigorous guide as to how this project will proceed. The 
first step was to figure out the particular classes of violations and the most 
conventional ways they are triggered. From there the developers of Ampersand 
have provided a partial algorithm for how individual violations can be 
restored. Foreseeable issues include not having a complete algorithm to 
implement on complex system such as Ampersand, in addition to figuring out how 
to fix violations that may cycle within itself.

\end{document}










