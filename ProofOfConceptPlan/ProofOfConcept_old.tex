\documentclass[12pt]{article}
\usepackage{nopageno}
\usepackage{xcolor} % for different colour comments
\usepackage{parskip} % Space between each paragraph.
\usepackage{hardwrap} % for text length of 80 pts
\usepackage[margin=1.2in]{geometry}
\usepackage{../ltx/edcomms}
\usepackage{titling}

\usepackage{ifthen}
\usepackage{../ltx/edcomms}

%% Comments are enabled and disabled by 'draft' mode. I hacked in my own draft
%% mode (https://en.wikibooks.org/wiki/LaTeX/Macros) because the LaTeX draft
%% mode disables a bunch of things that I don't want it to. I just want it to
%% disable comments. Do not set any of this manually, just use the build script,
%% which builds both draft and final copies. Comments are enabled by default, so
%% if you build manually, you get a draft copy. 
\providecommand\draftmode{true}

\ifthenelse{\equal{\draftmode}{true}}{
\newcommand{\authornote}[3]{\textcolor{#1}{[#3 ---#2]}}
\newcommand{\todo}[1]{\textcolor{red}{[TODO: #1]}}
%\edcommstrue %% Dr. Kahl's comment package. Eventually we should migrate all
             %% comments to this.
}{
\edcommsfalse 
\newcommand{\authornote}[3]{}
\newcommand{\todo}[1]{}
}

% wss = Dr. Smith ; ds = Dr. Szymczak
\newcommand{\wss}[1]{\authornote{magenta}{SS}{#1}}
\newcommand{\ds}[1]{\authornote{blue}{DS}{#1}}



\setlength{\parindent}{13pt} % parskip sets this to 0. 15 is default.
\setlength{\droptitle}{-9em}

\begin{document}

\title{CS4ZP6 Proof of Concept demo Plan \\ Ampersand Tarski Event-Condition-Action 
Rules \\ \vspace{2ex}} 
\author{\normalsize{Yuriy Toporovskyy,\ Yash Sapra,\ Jaeden Guo}}
\date{\normalsize\today \vspace{1ex}}
\thispagestyle{empty}
\maketitle
\paragraph{}
ASDSA
The focus of the test plan will be to demonstrate the current state of the 
Ampersand model prior to the addition of our project and to provide an overview 
of where our project fits into the Ampersand system. Ampersand is a system with 
patches for its unfinished pieces, and one of those missing pieces is the 
purpose of our project, EFA. Although Ampersand is functional, it is limited in 
the number of finished products and complete services it can provide to its 
users. This demonstration will be a black box test of the current Ampersand 
system, one designed to produce various kinds of errors and violations. The 
error messages inform the user of syntax and definition errors, such as a 
missing variable. Violations inform the user when the model they are attempting 
to build violates pre-set conditions within the Ampersand system. 
\newline\newline
\indent We will first compile a simple ADL script in Ampersand to generate a working
information system. At this point, once we have a working information system generated from Ampersand,
it will be easy to understand how the ADL script translates to a model and how by creating violations in the ADL script we can generate errors. After this we will partly go over the other supporting artifacts generated from Ampersand and explain how these are useful from a business point of view. 
Once the audience is aware of the capabilities of Ampersand, we will be reviewing 
the ADL script to establish that the ADL script is just plain text written in Natural Language that
 describes the system to be.
We will explain how the real world constraints are reflected in the ADL script and in the information system produced by running that script. Further some violations will be inserted in the ADL script to generate error messages to show what type of violations Ampersand is capable of handling. Once the user is familiar with Ampersand and the artifacts it generates we'll get into the ``how'' of  Ampersand; explaining the ADL script we used. 
 
 At this point, the audience will have a good understanding of what Ampersand does, what type of artifacts it is capable of producing and the syntax and semantics of the ADL file. With all the prerequisites in place, we will go ahead and explain where our project (EFA) fits in, into the Ampersand system and how our contribution will change the behavior of Ampersand. The purpose of our project is to automate the handling of data violations; we will explain what these violations are, how the current system reacts to these violations and how our project will change the way Ampersand will handle these violations.
 
%The focus of the test plan will be to demonstrate the current state of 
%Ampersand model prior to the addition of our project and to provide an overview 
%of where our project fits into the Ampersand system. Ampersand in its current state is able 
%to compile an ADL script and generate a working prototype and supporting artifacts on successful compilation of code.
%\newline
%\indent We will first compile a simple ADL script in Ampersand to generate a working
%information system. At this point, once we've a working information system generated from Ampersand,
%it will be easy to understand how the ADL script translates to a model and how by creating violations in the ADL script we can generate errors. After this we'll partly go over the other supporting artifacts generated from Ampersand, lik
%Once the audience is aware of the capabilities of Ampersand, we'll be reviewing 
%the ADL script to establish that the ADL script is just plain text written in Natural Language that
% describes the system to be.
% We'll explain how the real world constraints are reflected in the ADL script and the information system produced by running that script. Further some violations will be inserted in the ADL script to generate error messages to show what type of violations Ampersand is capable of handling. Once the user is familiar with Ampersand and the artifacts it generates we'll get into the ``how'' of  Ampersand; explaining the ADL script we used. 
% 
% At this point the audience will have a good understanding of what Ampersand does, what type of artifacts it is capable of producing and the syntax and semantics of the ADL file. With all the prerequisites in place, we will go ahead and explain where our project (EFA) fits in, into the Ampersand system and how our contribution will change the behavior of Ampersand. The purpose of our project is to automate correction of violations in the system, we'll explain what these violations are, how the current system reacts to these violations and how our project will change the way Ampersand will handle these violations.
 
% The purpose of the EFA project is to automate the correction of a 
%particular class of violations in order to restore invariants within the 
%Ampersand system. These violations come from the misuse of PAClauses (Process 
%Algebra Clauses); PAClauses are datatypes formed from ECA (Event-Condition 
%Action) rules that are used to represent the structure of an active database 
%system. In Ampersand, these ECA rules are deconstructed to their most basic 
%forms which we call “Atoms”. Atoms are the simplest possible form a rule can 
%take before losing its meaning. From here, a Haskell program (i.e., EFA) will 
%automate the correction of violations by detecting contradictions between Atoms 
%imposed on various datasets. For example, one Atom states that all entities in 
%its set must be blue, another Atom states that all entities in its set must be 
%red. The intersection between these two Atoms is red AND blue; if the user 
%inserts something in the intersection of red and blue that is not red and blue, 
%it is a violation of both Atoms. A quick fix to restore both rules would be to 
%remove the entity that is not red or blue from the intersection of the red and 
%blue set. 
%\newline
%\indent EFA is in its initial stages of development, and at this time it is ``
%restored. Foreseeable issues include not having a complete algorithm to 
%implement on complex system such as Ampersand, in addition to figuring out how 
%to fix violations that may cycle within itself.

\end{document}










