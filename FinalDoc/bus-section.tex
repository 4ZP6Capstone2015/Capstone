%\documentclass[document.tex]{subfiles}
%\begin{document}

Ampersand follows a \em{rule based} design principle. Rules are integral to an organization
and these are based on some principles and guidelines set by an organization.
Ampersand uses an ECA ( Event - Condition - Action) approach to make sure all rules are satisfied. An ideal information infrastructure supports employees and other stakeholders to maintain the rules of the business. To maintain a rule means to prevent or correct all violations that might occur due to any external or internal factor.
 
 This problem is addressed in Ampersand using AMMBR \citep{Ampersand}. The role of the AMMBR method in Ampersand is to maintain these business rules. In AMMBR, human involvement is only limited to representing rules (in the ADL files). While the Ampersand software still remains in development phase, there is no way of checking the correctness of the AMMBR method at a low level. Previously the ECA rule generated by AMMBR was directly fed into the Exec Engine to maintain these business rules. 
 
With the extension of Ampersand by the EFA project, we allow means of checking the correctness of the Action associated with an ECA rule. The ECA rules, which act as an input to the EFA project are translated to human-readable SQL. This can later be viewed in the command line using the \verb|``~-~-print~-eca~-info''| flag written out in one of the artifacts. The generated SQL, not only allows for checking the correctness of AMMBR but also helps in identifying 
patterns and unreachable states of the method. EFA project will utilized to complete the AMMBR algorithm and also to test out future modification to the AMMBR algorithm. Since AMMBR is an essential element of the Ampersand software, EFA will compliment future modification to this method.
 

%\end{document}