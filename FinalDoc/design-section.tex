%\documentclass[document.tex]{subfiles}
\noindent
\section{System Architecture and Module Hierarchy}
\subsection{System Architecture}\label{SystemArch}
This section provides an overview of system architecture and module hierarchy. 
The 
initial section introduces term and tools used in the making of each EFA 
module. The module design is detailed with UML-like class diagrams. However, 
UML class diagrams are typically
used to describe the module systems of object-oriented programs, as opposed to
functional programs. Many of the components of the traditional UML class
diagram are inapplicable to functional programs; therefore, we detail our
modifications to the UML class diagram syntax in 
section%~\ref{subsec:ModuleSyntax}. 

Furthermore, the syntax used to describe types and data declarations is not
actual Haskell syntax. The syntax shares many similarities, but several changes
to the syntax are made in this document in order to present the module hierarchy
in a clear manner. These changes are also detailed, in 
section%~\ref{subsec:HaskellSyntax}. 

\begin{figure}[!ht]
    \makebox[\textwidth][c]{
        \scalebox{0.6}{
            \begin{tikzpicture}
            
            \begin{package}{Module}
            
            \begin{class}[text width=10cm]{Submodule$_0$}{0,0}
            \hstype{Data$_0$ : Type where}
            \hstypectr{Ctr$_0$ : A -> Data$_0$}
            \hstypectr{pattern Ctr$_1$ : B -> Data$_0$}
            \hsfunc{func$_0$ : B -> C -> D} 
            \hstype{Data$_1$ : Kind where}
            \hstypectr{Ctr$_2$ : E -> Data$_1$}
            \end{class}
            
            \begin{class}[text width=10cm]{Submodule$_1$}{0,-5}
            \attribute{\emph{Type level namespace}}
            \operation{\emph{Value level namespace}}
            \end{class}
            
            \draw [umlcd style, ->] (Submodule$_0$.south) -- (Submodule$_0$ |- 
            Submodule$_1$.north); 
            \end{package}
            
            \begin{class}[text width=3cm]{Module$_0$}{10,-3.5}
            \end{class}
            
            \begin{class}[text width=3cm]{Module$_1$}{10,-5.5}
            \end{class}
            
            \begin{class}[text width=3cm]{Module$_2$}{15,-5.5}
            \end{class}
            
            \draw [umlcd style, ->] (Module$_0$.south) -- (Module$_0$ |- 
            Module$_1$.north); 
            \draw [umlcd style, ->] (Module$_1$.east) -- (Module$_1$ -| 
            Module$_2$.west); 
            
            \renewcommand{\umlfillcolor}{grey}
            \begin{class}[text width=10cm]{Package Dependency}{12,0.5}
            \hstype{$\ldots$}
            \hsfunc{$\ldots$}
            \end{class}
            
            
            \renewcommand{\umlfillcolor}{applegreen}
            \begin{class}[text width=10cm]{System Module}{12,-1.5}
            \hstype{$\ldots$}
            \hsfunc{$\ldots$}
            \end{class}
            
            \end{tikzpicture}
        }}\caption{Example of module diagram syntax}\label{fig:ModExample}
    \end{figure}



\noindent
\subsection{A Description of Haskell-Similar Syntax}\label{subsec:HaskellSyntax}

This section details the syntax used to describe the module system of
Ampersand. This syntax largely borrows from actual Haskell syntax, and from the
Agda programming language%\cite{agda}.
 Agda is a dependently typed functional
language, and since a large part of our work deals with ``faking'' dependent
types, the syntax of Agda is conducive to easy communication of our module 
system. The
principle of faking dependent types in Haskell is detailed in
Hasochism \citep{hasochism} 
(a portmanteau of Haskell and masochism, because
purportedly wanting to fake dependent types in Haskell is masochism). While the
implementation has since been refined many times over, the general approach is 
still the
same, and will not be detailed here.
While the changes made to the Haskell syntax are reasonably complex, the 
ensuing 
module description becomes vastly simplified. This section is meant to be used
as a reference - in many cases, the meaning of a type is self-evident. 

\noindent
\subsection{Description of Types and Kinds}
In the way that a type classifies a set of values, a kind classifies a set of
types. Haskell permits one to define algebraic data types, which are then 
``promoted''
to the kind level \citep{promotion}. 
This permits the type constructor of the datatype to be used
as a kind constructor, and for the value constructors to be used as type 
constructors. In every case in our system, when we define a datatype and use 
the promoted version, we never use the \emph{unpromoted} version. That is, we 
define types which are never used as types, only as kinds, and constructors 
which are never used as value constructors,only type constructors. We write 
\,\,\,\lstinline!X : A -> B -> $\ldots$ -> Type!\,\,\, to denote
a regular data type, and \,\,\,\lstinline!Y : A -> B -> $\ldots$ -> Kind!\,\,\, 
to denote a datatype
which is used exclusively as a kind. 

\noindent
\subsection{Description of Dependant types}
The syntax used to denote a ``fake'' dependent type in our model is the same 
as used to denote a real dependent type in Agda. \lstinline!(x : A) -> B! is 
the function
from $x$ to some value of type $B$, where $B$ can mention $x$. This nearly 
looks like a 
real Haskell type - in Haskell, the syntax would be \texttt{forall (x :: A) . 
B}. However, 
the semantics of these two types are vastly different - the former can pattern 
match
on the value of $x$, while the latter cannot. 

In certain cases, it may be elucidating to see the \emph{real} Haskell type of
an entity (function, datatype, etc.). To differentiate the two, they are typeset
differently, as in this example.

The real type of a function whose type is given as \lstinline!(x : A) -> B! in
our model is \texttt{forall (x :: A) . SingT x -> B}. \texttt{SingT :: A -> 
Type}
denotes the singleton type for the kind $A$, which is inhabitted by precisely
one value for each type which inhabits $A$. The role and use of singleton types
is detailed further on, in section~\ref{subsec:Singletons}. 

The syntax \lstinline!forall (x : A) -> B! is used to denote the regular
Haskell type \texttt{forall (x :: A) . B}. As is customary in Haskell, the 
quantification
may be dropped when the kind $A$ is clear from the context: 
\lstinline!forall (x : A) -> P x! and
\lstinline!forall x -> P x! denote the type \texttt{forall (x :: A) . P x}.

\subsection{Constraints}
The Haskell syntax \texttt{A -> B} denotes a function from $A$ to $B$. However,
we use the arrow to additionally denote constraints. For example, the function
\texttt{Show a => a -> String} would be written simply as 
\lstinline!Show a -> a -> String!.
In certain casees, a constraint is intended to be used only in an implicit 
fashion 
(i.e. as an actual constraint), in which case the constraint is written with 
the typical \lstinline{=>} syntax. 

\subsubsection{Existential quantification}
The type \lstinline!exists (x : A) (P x)! indicates that there exists some $x$
of kind $A$ which satisfies the predicate $P$. Unfortunately, Haskell does not
have first class existential quantification. It must be encoded in one of
two ways:

\subsubsection{Types, kinds, and type synonyms}
Type synonyms are written in the model as \lstinline!Ty : K = X!, where $Ty$ is 
the name
of the type synonym, $K$ is its kind, and $X$ its implementation. This is to 
differentiate
from type families, which are written as 
\lstinline!Ty : K where Ty $\ldots$ = $\ldots$!.

\subsubsection{Overloading}
Haskell supports overloaded function names through type classes. When we use a 
type 
class to simply overload a function name, we simply write the function name
multiple times with different types. The motivation for this is that often the 
real type will be exceeding complex, because it must be so to get good type 
inference. 

 

\subsubsection{Omitted implementations}
When the implementation of a type synonym, or any other entity, is omitted, it
is replaced by ``$\ldots$''. This is to differentiate from a declaration of the 
form
\lstinline!Ty : Type!, which is an abstract type whose constructors cannot be
accessed. Furthermore, types may have pattern-match-only constructors; that is,
constructors which can only be used in the context of a pattern match, and not
to construct a value of that type. This is denoted by the syntax
``\lstinline!pattern Ctr : Ty!''. Furthermore, it is not a simple matter of
convention - the use of this constructor in expressions will be strictly
forbidden by Haskell.


\begin{itemize}
    \item With a function (by DeMorgan's law): \\ \texttt{(forall (x :: A) . P 
    x -> r) -> r}
    \item With a datatype: \\ \texttt{data Exists p where Exists :: p x -> 
    Exists p}
\end{itemize} 

Choice of form is based on the circumstances in which the function
will most likely be used, since whether one form is more convenient than the
other depends largely on the intended use. However, these two forms are
completely interchangeable (albeit with some syntactic noise) so the syntax
presented here does not distinguish between the two. 

\section{Module Hierarchy}
\subsection{A Description of Module Diagram Syntax}\label{subsec:ModuleSyntax}

The module hierarchy is broken down into multiple levels to better describe the
system.  A coarse module hierarchy is given, and each module is further broken
into submodules.  A dependency between two modules $A$ and $B$ indicates that
each submodule in $A$ depends on all of $B$. There is no necessity to break
down modules into submodule, if they do not have any interesting submodule 
structure. Arrows between modules and submodules denote a dependency. 

External dependencies, which are modules which come from an external pacakge,
are indicated in {\color{grey}grey}. System modules, which are modules part of
Ampersand, but not written specifically for EFA (or, on which EFA depends, but
few or no changes have been made from the original module before the existance
of EFA), are indicated in {\color{applegreen}green}. The module heirarchy of
these modules is not described here; they are included simply to indicate which
symbols are imported from these modules. An example of the syntax is found in
figure%~\ref{fig:ModExample}.

%----------------------------------------------------------
This section contains a hierarchal breakdown of each module, as well as a brief
explanation of each modules' elements. 

The module heiarchy of EFA as a whole is given in figure~\ref{fig:efaMod}.  Note
that every module which is part of EFA depends on the Haskell \texttt{base} 
package
(which is the core libraries of Haskell). Also note that for the \texttt{base}
package, we only include primitive definitions (i.e. those not defined in real
Haskell) which may be difficult to track down in the documentation. The kinds
\lstinline{Nat} and \lstinline{Symbol} correspond to type level natural number 
and string
literals, respectively. The kind \lstinline{Constraint} is the kind of class and
equality constraints, for example, things like \lstinline{Show x} and 
\lstinline[mathescape]|Int $\sim$ Bool|.  
Note that \texttt{Show} itself does \emph{not} have kind Constraint --
its kind is \lstinline{Type -> Constraint}. The detailed semantics of these
primitive entities can be found in the GHC user guide\citep{ghcUserGuide}. While
many modern features of GHC are used in the actual implementation, they are not
mentioned in, nor required to understand, the module description.
%% All other features of GHC are
%% detailed in the user guide as well, but no other

The primary interface to EFA is the function \lstinline{eca2PrettySQL}, which 
takes an
FSpec (the abstract syntax of Ampersand) and an ECA rule, and returns the pretty
printed SQL code for that rule. Also note that while the dependencies within EFA
modules is relatively complex, they depend on the rest of the Ampersand system
in a simple manner. The modules \lstinline{Test} and \lstinline{Prototype} 
implement the testing
framework and the prototype generation, respectively; these modules depend
directly on only one module from EFA, namely \lstinline{ECA2SQL}. Similarly, the
majority of EFA itself does not depend directly on Ampersand modules outside of
EFA. This makes EFA very resilient to changes in the core Ampersand system; in
order to update EFA to work with a modification to Ampersand, only one EFA
module -- ECA2SQL -- will generally need to be modified. 

All functions named in the module hierarchy are total - they do not throw
exceptions, or produce unhandled errors or infinite loops. Therefore, no
additional information past the type of the function is required to deduce the
inputs and ouputs of the function -- they are precisely the inputs and outputs
of the type.

\begin{figure}[!ht]
    \makebox[\textwidth][c]{
        \scalebox{0.6}{
            \begin{tikzpicture}
            
            %% \begin{package}{TypedSQL}
            %% fwQcGWGGF
            
            \begin{class}[text width=16.6cm]{ECA2SQL}{0,2}
            \hsfunc{eca2SQL : FSpec -> ECArule -> exists (a : [SQLType]) 
            (SQLMethod a SQLBool)}
            \hsfunc{eca2PrettySQL : FSpec -> ECArule -> Doc}
            %% \hsfunc{$\ldots$}
            \end{class}
            
            \begin{class}[text width=3cm]{TypedSQL}{-3.4, -2.5}
            \hsfunc{$\ldots$}
            \end{class}
            
            \begin{class}[text width=4.5cm]{PrettyPrinterSQL}{-9, -2.5}
            \hsfunc{$\ldots$}
            \end{class}
            
            \begin{class}[text width=6cm]{TypedSQLCombinators}{5,-2.5}
            \hsfunc{$\ldots$}
            \end{class}
            
            
            \begin{class}[text width=3cm]{Equality}{-3.4,-5}
            \hsfunc{$\ldots$}
            \end{class}
            
            \begin{class}[text width=3cm]{Singletons}{3.55,-5}
            \hsfunc{$\ldots$}
            \end{class}
            
            
            %% \begin{class}[text width=3cm]{Trace}{-3.4,-13.5}
            %%     \hsfunc{$\ldots$}
            %% \end{class}
            
            \begin{class}[text width=3cm]{Proof Utils}{3.55,-7.5}
            \hsfunc{$\ldots$}
            \end{class}
            
            \renewcommand{\umlfillcolor}{grey}
            \begin{class}[text width=9cm]{base}{6,-13.5}
            \hstype{Nat : Kind} 
            \hstype{Constraint : Kind} 
            \hstype{Type : Kind} 
            \hstype{Symbol : Kind}
            \hstype{CmpSymbol : Symbol -> Symbol -> Ordering} 
            \hstype{(\~) : Type -> Type -> Constraint} 
            \end{class}
            
            \renewcommand{\umlfillcolor}{grey}
            \begin{class}[text width=6cm]{wl-pprint}{-3.4,-6.5}
            \hstype{Doc : Type} 
            \hstype{instance Show Doc where $\ldots$} 
            \hstype{class Pretty a where}
            \hstype[20pt]{pretty :: a -> Doc}
            \hsfunc{$\ldots$}
            \end{class}
            
            \renewcommand{\umlfillcolor}{grey}
            \begin{class}[text width=7cm]{simple-sql-parser}{-11,-6.5}
            \hstype{QueryExpr : Type where $\ldots$}
            \hstype{ValueExpr : Type where $\ldots$}
            \hstype{Name : Type where $\ldots$}
            \hsfunc{prettyQueryExpr : QueryExpr -> Doc} 
            \hsfunc{prettyValueExpr : ValueExpr -> Doc} 
            \end{class}
            
            
            \renewcommand{\umlfillcolor}{applegreen}
            \begin{class}[text width=4.9cm]{AmpersandCore}{4,-10}
            \hstype{ECArule : Type where $\ldots$} 
            %% \hsfunc{$\ldots$}
            \end{class}
            
            \renewcommand{\umlfillcolor}{applegreen}
            \begin{class}[text width=3cm]{Test}{-2,5}
            \hsfunc{$\ldots$}
            \end{class}
            
            \renewcommand{\umlfillcolor}{applegreen}
            \begin{class}[text width=3cm]{Prototype}{2,5}
            \hsfunc{$\ldots$}
            \end{class}
            
            \renewcommand{\umlfillcolor}{applegreen}
            \begin{class}[text width=4.5cm]{FSpec}{10.4,-10}
            \hstype{FSpec : Type where $\ldots$} 
            %% \hsfunc{$\ldots$}
            \end{class}
            
            \unidirectionalAssociation{Prototype}{}{}{ECA2SQL}
            \unidirectionalAssociation{Test}{}{}{ECA2SQL}
            
            \draw [umlcd style, ->] ($(ECA2SQL.south)!0.5!(ECA2SQL.south 
            west)$) -- (TypedSQL.north); 
            \draw [umlcd style, ->] ($(ECA2SQL.south)!0.5!(ECA2SQL.south 
            east)$) -- (TypedSQLCombinators.north); 
            \draw [umlcd style, ->] (ECA2SQL.south) -- (Equality.north east); 
            \draw [umlcd style, ->] (ECA2SQL.south) -- (Singletons.north west); 
            %% \draw [umlcd style, ->] (ECA2SQL.south) -- (Trace.north east); 
            \draw [umlcd style, ->] (ECA2SQL.south) -- (Proof Utils.north 
            west); 
            \draw [umlcd style, ->] (ECA2SQL.south west) -- 
            (PrettyPrinterSQL.north); 
            
            
            \node[above = 1cm of base] (basedummy)       {};
            \node[above = 1.5cm of base](basedummy2)       {};
            \draw [umlcd style, ->, thick] (basedummy) -- (base); 
            \draw [umlcd style dashed line, -, thick] (basedummy2) -- (base); 
            
            \draw [umlcd style, ->] (Singletons.south) -- (Proof Utils.north); 
            %% \draw [umlcd style, ->] (Singletons.west) -- (Trace.east); 
            \draw [umlcd style, ->] (TypedSQL.south) -- (Equality.north); 
            %% \draw [umlcd style, ->] (Equality.south) -- (Trace.north); 
            \draw [umlcd style, ->] (TypedSQLCombinators.west) -- 
            (TypedSQL.east); 
            \draw [umlcd style, ->] (Singletons.west) -- (Equality.east); 
            %% \draw [umlcd style, ->] (Proof Utils.south west) -- (Trace.south 
            %%east); 
            \draw [umlcd style, ->] (TypedSQL.south east) -- 
            (AmpersandCore.north west); 
            
            \draw [umlcd style, ->] ([xshift=60pt]TypedSQLCombinators.south) -- 
            ([xshift=30pt]AmpersandCore.north); 
            \draw [umlcd style, ->] (TypedSQLCombinators.south) -- 
            (Singletons.north); 
            \draw [umlcd style, ->] ([xshift=30pt]TypedSQLCombinators.south) -- 
            (Proof Utils.north east); 
            
            
            \draw [umlcd style, ->] ([xshift=-30pt]PrettyPrinterSQL.south east) 
            -- (wl-pprint.north west); 
            %% \draw [umlcd style, ->] (PrettyPrinterSQL.south east) -- 
            %%(Trace.west); 
            \draw [umlcd style, ->] (PrettyPrinterSQL.east) -- (TypedSQL.west); 
            
            %% \node[above = 1cm of Trace] (tracedummy)       {};
            %% \node[above = 1.5cm of Trace](tracedummy2)       {};
            %% \draw [umlcd style, ->, thick] (tracedummy) -- (Trace); 
            %% \draw [umlcd style dashed line, -, thick] (tracedummy2) -- 
            %%(Trace); 
            
            \draw [umlcd style, ->] (TypedSQL.south west) -- 
            (simple-sql-parser.north east); 
            
            \draw  [umlcd style, ->, fill opacity=0]  
            ([xshift=70pt]TypedSQLCombinators.south) --++ (0cm,-8cm) -| 
            (simple-sql-parser.south);
            
            
            \draw[umlcd style ,->, fill opacity=0] (ECA2SQL.east) -| node[above 
            , sloped , black]{} (FSpec.north);
            
            \draw[umlcd style ,->, fill opacity=0] (ECA2SQL.south) |- 
            node[above , sloped , black]{} (AmpersandCore.west);
            
            \unidirectionalAssociation{PrettyPrinterSQL}{}{}{simple-sql-parser}
            
            \end{tikzpicture}
        }}\caption{Module diagram for EFA as a whole} \label{fig:efaMod}
    \end{figure}
    

