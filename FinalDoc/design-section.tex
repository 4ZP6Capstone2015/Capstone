%\documentclass[document.tex]{subfiles}
\noindent
\section{System Architecture and Module Hierarchy}

\subsection{System Architecture}\label{SystemArch}

This section provides an overview of system architecture and module hierarchy. 
The 
initial section introduces term and tools used in the making of each EFA 
module. The module design is detailed with UML-like class diagrams. However, 
UML class diagrams are typically
used to describe the module systems of object-oriented programs, as opposed to
functional programs. Many of the components of the traditional UML class
diagram are inapplicable to functional programs; therefore, we detail our
modifications to the UML class diagram syntax in 
section ~\ref{subsec:ModuleSyntax}. 

Furthermore, the syntax used to describe types and data declarations is not
actual Haskell syntax. The syntax shares many similarities, but several changes
to the syntax are made in this document in order to present the module hierarchy
in a clear manner. These changes are also detailed, in 
section ~\ref{subsec:HaskellSyntax}. 

\subsection{External Libraries}
\noindent No addition dependencies are required outside of those that Ampersand 
already uses. EFA directly uses the following dependencies of Ampersand:
\begin{description}
    \item[Ampersand Core Libraries]
    The EFA project depends on the Ampersand software for the definition of 
    core Data Structures, (i.e. FSpec, which contains the definition of 
    the underlying ECA rules). EFA also maintains the relational schema of 
    the input, and hence, imports Ampersand's existing functions to fetch 
    the table declarations while generating SQL Statements for the ECA 
    rules. AMMBR \cite{AMMBR}, which is the key algorithm responsible for 
    translating business requirements into ECA rules is an integral part of 
    Ampersand.
    \item[simple-sql-parser]
    EFA's pretty printer depends directly on this library for formatting 
    and printing SQL statements. The SQL statement syntax 
    defined here is built on top of the existing expression syntax defined 
    in this package. This package is the one used by the core Ampersand system,
    so our use of it facilitates interaction and integration with Ampersand. 
    \cite{simple-sql}
    \item[wl-pprint] 
    The wl-pprint library\cite{wl-pprint} is a pretty printer based on the
    pretty printing combinators. EFA uses this library in combination with
    the simple-sql-pretty to output the SQL statements in a human readable
    format.
    \item[deepseq]
    The deepseq library provides a type class which implements a function
    for reducing values to normal formal, similarly to the built in
    \lstinline{seq} function.
\end{description}

\noindent
\subsection{A Description of Haskell-Like Syntax}\label{subsec:HaskellSyntax}

This section details the syntax used to describe the module system of
Ampersand. This syntax largely borrows from actual Haskell syntax, and from the
Agda programming language~\cite{agda}.
Agda is a dependently typed functional
language, and since a large part of our work deals with ``faking'' dependent
types, the syntax of Agda is conducive to easy communication of our module 
system. The
principle of faking dependent types in Haskell is detailed in
Hasochism \citep{hasochism} 
(a portmanteau of Haskell and masochism, because
purportedly wanting to fake dependent types in Haskell is masochism). While the
implementation has since been refined many times over, the general approach is 
still the
same, and will not be detailed here.
While the changes made to the Haskell syntax are reasonably complex, the 
ensuing 
module description becomes vastly simplified. This section is meant to be used
as a reference - in many cases, the meaning of a type is self-evident. 

\noindent
\subsubsection*{Description of Types and Kinds}

In the way that a type classifies a set of values, a kind classifies a set of
types. Haskell permits one to define algebraic data types, which are then 
``promoted''
to the kind level \citep{promotion}. 
This permits the type constructor of the datatype to be used
as a kind constructor, and for the value constructors to be used as type 
constructors. In every case in our system, when we define a datatype and use 
the promoted version, we never use the \emph{unpromoted} version. That is, we 
define types which are never used as types, only as kinds, and constructors 
which are never used as value constructors, only type constructors. We write 
\,\,\,
\lstinline!X : A -> B -> $\ldots$ -> Type!
\,\,\, 
to denote a regular data type, and 
\,\,\,
\lstinline!Y : A -> B -> $\ldots$ -> Kind!
\,\,\, 
to denote a datatype
which is used exclusively as a kind. 

\noindent
\subsubsection*{Description of Dependant types}

The syntax used to denote a ``fake'' dependent type in our model is the same 
as used to denote a real dependent type in Agda. \lstinline!(x : A) -> B! is 
the function
from $x$ to some value of type $B$, where $B$ can mention $x$. This nearly 
looks like a 
real Haskell type - in Haskell, the syntax would be \texttt{forall (x :: A) . 
B}. However, 
the semantics of these two types are vastly different - the former can pattern 
match
on the value of $x$, while the latter cannot. 

In certain cases, it may be elucidating to see the \emph{real} Haskell type of
an entity (function, datatype, etc.). To differentiate the two, they are typeset
differently, as in this example.

The real type of a function whose type is given as \lstinline!(x : A) -> B! in
our model is \texttt{forall (x :: A) . SingT x -> B}. \texttt{SingT :: A -> 
Type}
denotes the singleton type for the kind $A$, which is inhabited by precisely
one value for each type which inhabits $A$. The role and use of singleton types
is detailed further on, in section~\ref{subsec:Singletons}. 

The syntax \lstinline!forall (x : A) -> B! is used to denote the regular
Haskell type \texttt{forall (x :: A) . B}. As is customary in Haskell, the 
quantification
may be dropped when the kind $A$ is clear from the context: 
\lstinline!forall (x : A) -> P x! 
and
\lstinline!forall x -> P x! denote the type \texttt{forall (x :: A) . P x}.

\subsubsection*{Constraints}

The Haskell syntax \texttt{A -> B} denotes a function from $A$ to $B$. However,
we use the arrow to additionally denote constraints. For example, the function
\texttt{Show a => a -> String} would be written simply as 
\lstinline!Show a -> a -> String!
.
In certain cases, a constraint is intended to be used only in an implicit 
fashion 
(i.e. as an actual constraint), in which case the constraint is written with 
the typical \lstinline{=>} syntax. 

\subsubsection*{Existential quantification}

The type \lstinline!exists (x : A) (P x)! indicates that there exists some $x$
of kind $A$ which satisfies the predicate $P$. Unfortunately, Haskell does not
have first class existential quantification. It must be encoded in one of
two ways:


\begin{itemize}
    \item With a function (by DeMorgan's law): \\ \texttt{(forall (x :: A) . P 
    x -> r) -> r}
    \item With a datatype: \\ \texttt{data Exists p where Exists :: p x -> 
    Exists p}
\end{itemize} 

Which form is used is decided based on the circumstances in which the function
will most likely be used, since whether one form is more convenient than the
other depends largely on the intended use. However, these two forms are
completely interchangeable (albeit with some syntactic noise) so the syntax
presented here does not distinguish between the two. 

\subsubsection*{Types, kinds, and type synonyms}

Type synonyms are written in the model as \lstinline!Ty : K = X!, where $Ty$ is 
the name
of the type synonym, $K$ is its kind, and $X$ its implementation. This is to 
differentiate
from type families, which are written as 
\lstinline!Ty : K where Ty $\ldots$ = $\ldots$!.

\subsubsection*{Overloading}

Haskell supports overloaded function names through type classes. When we use a 
type 
class to simply overload a function name, we simply write the function name
multiple times with different types. The motivation for this is that often the 
real type will be exceeding complex, because it must be so to get good type 
inference. 



\subsubsection*{Omitted implementations}

When the implementation of a type synonym, or any other entity, is omitted, it
is replaced by ``$\ldots$''. This is to differentiate from a declaration of the 
form
\lstinline!Ty : Type!, which is an abstract type whose constructors cannot be
accessed. Furthermore, types may have pattern-match-only constructors; that is,
constructors which can only be used in the context of a pattern match, and not
to construct a value of that type. This is denoted by the syntax
``\lstinline!pattern Ctr : Ty!''. It is not a simple matter of
convention - the use of this constructor in expressions will be strictly
forbidden by Haskell.


\begin{figure}[!ht]
\makebox[\textwidth][c]{
\scalebox{0.6}{
\begin{tikzpicture}

\begin{package}{Module}

\begin{class}[text width=10cm]{Submodule$_0$}{0,0}
\hstype{Data$_0$ : Type where}
\hstypectr{Ctr$_0$ : A -> Data$_0$}
\hstypectr{pattern Ctr$_1$ : B -> Data$_0$}
\hsfunc{func$_0$ : B -> C -> D} 
\hstype{Data$_1$ : Kind where}
\hstypectr{Ctr$_2$ : E -> Data$_1$}
\end{class}

\begin{class}[text width=10cm]{Submodule$_1$}{0,-5}
\attribute{\emph{Type level namespace}}
\operation{\emph{Value level namespace}}
\end{class}

\draw [umlcd style, ->] (Submodule$_0$.south) -- (Submodule$_0$ |- 
Submodule$_1$.north); 
\end{package}

\begin{class}[text width=3cm]{Module$_0$}{10,-3.5}
\end{class}

\begin{class}[text width=3cm]{Module$_1$}{10,-5.5}
\end{class}

\begin{class}[text width=3cm]{Module$_2$}{15,-5.5}
\end{class}

\draw [umlcd style, ->] (Module$_0$.south) -- (Module$_0$ |- 
Module$_1$.north); 
\draw [umlcd style, ->] (Module$_1$.east) -- (Module$_1$ -| 
Module$_2$.west); 

\renewcommand{\umlfillcolor}{grey}
\begin{class}[text width=10cm]{Package Dependency}{12,0.5}
\hstype{$\ldots$}
\hsfunc{$\ldots$}
\end{class}


\renewcommand{\umlfillcolor}{applegreen}
\begin{class}[text width=10cm]{System Module}{12,-1.5}
\hstype{$\ldots$}
\hsfunc{$\ldots$}
\end{class}

\end{tikzpicture}
}}\caption{Example of module diagram syntax}\label{fig:ModExample}
\end{figure}



\subsection{A Description of Module Diagram Syntax}\label{subsec:ModuleSyntax}

The module hierarchy is broken down into multiple levels to better describe the
system.  A coarse module hierarchy is given, and each module is further broken
into submodules.  A dependency between two modules $A$ and $B$ indicates that
each submodule in $A$ depends on all of $B$. There is no necessity to break
down modules into submodule, if they do not have any interesting submodule 
structure. Arrows between modules and submodules denote a dependency. 

External dependencies, which are modules which come from an external package,
are indicated in {\color{grey}grey}. System modules, which are modules part of
Ampersand, but not written specifically for EFA (or, on which EFA depends, but
few or no changes have been made from the original module before the existence
of EFA), are indicated in {\color{applegreen}green}. The module hierarchy of
these modules is not described here; they are included simply to indicate which
symbols are imported from these modules. An example of the syntax is found in
figure%~\ref{fig:ModExample}.

\subsection{Module Hierarchy}\label{subsec:modhierarchy}

This section contains a hierarchal breakdown of each module, as well as a brief
explanation of each modules' elements. The module hierarchy of EFA as a whole 
is 
given in figure~\ref{fig:efaMod}.  Note
that every module which is part of EFA depends on the Haskell \texttt{base} 
package
(which is the core libraries of Haskell). Also note that for the \texttt{base}
package, we only include primitive definitions (i.e. those not defined in real
Haskell) which may be difficult to track down in the documentation. The kinds
\lstinline{Nat} and \lstinline{Symbol} correspond to type level natural number 
and string
literals, respectively. The kind \lstinline{Constraint} is the kind of class and
equality constraints, for example, things like \lstinline{Show x} and 
\lstinline[mathescape]|Int $\sim$ Bool|.  
Note that \texttt{Show} itself does \emph{not} have kind Constraint --
its kind is \lstinline{Type -> Constraint}. The detailed semantics of these
primitive entities can be found in the GHC user guide~\cite{ghcUserGuide}. While
many modern features of GHC are used in the actual implementation, they are not
mentioned in, nor required to understand, the module description.

%% %% All other features of GHC are
%% %% detailed in the user guide as well, but no other

The primary interface to EFA is the function \lstinline{eca2PrettySQL}, which 
takes an FSpec (the abstract syntax of Ampersand) and an ECA rule, and returns 
the pretty
printed SQL code for that rule. Also note that while the dependencies within EFA
modules is relatively complex, they depend on the rest of the Ampersand system
in a simple manner. The modules \lstinline{Test} and \lstinline{Prototype} 
implement the testing
framework and the prototype generation, respectively; these modules depend
directly on only one module from EFA, namely \lstinline{ECA2SQL}. Similarly, the
majority of EFA itself does not depend directly on Ampersand modules outside of
EFA. This makes EFA very resilient to changes in the core Ampersand system; in
order to update EFA to work with a modification to Ampersand, only one EFA
module -- ECA2SQL -- will generally need to be modified. 

All functions named in the module hierarchy are total - they do not throw
exceptions, or produce errors which are not handled or infinite loops. 
Therefore, no
additional information past the type of the function is required to deduce the
inputs and outputs of the function -- they are precisely the inputs and outputs
of the type.

\begin{figure}[!ht]
\makebox[\textwidth][c]{
\scalebox{0.6}{
\begin{tikzpicture}

%% \begin{package}{TypedSQL}
%% fwQcGWGGF

\begin{class}[text width=16.6cm]{ECA2SQL}{0,2}
\hsfunc{eca2SQL : FSpec -> ECArule -> exists (a : [SQLType]) 
(SQLMethod a SQLBool)}
\hsfunc{eca2PrettySQL : FSpec -> ECArule -> Doc}
%% \hsfunc{$\ldots$}
\end{class}

\begin{class}[text width=3cm]{TypedSQL}{-3.4, -2.5}
\hsfunc{$\ldots$}
\end{class}

\begin{class}[text width=4.5cm]{PrettyPrinterSQL}{-9, -2.5}
\hsfunc{$\ldots$}
\end{class}

\begin{class}[text width=6cm]{TypedSQLCombinators}{5,-2.5}
\hsfunc{$\ldots$}
\end{class}


\begin{class}[text width=3cm]{Equality}{-3.4,-5}
\hsfunc{$\ldots$}
\end{class}

\begin{class}[text width=3cm]{Singletons}{3.55,-5}
\hsfunc{$\ldots$}
\end{class}


%% \begin{class}[text width=3cm]{Trace}{-3.4,-13.5}
%%     \hsfunc{$\ldots$}
%% \end{class}

\begin{class}[text width=3cm]{Proof Utils}{3.55,-7.5}
\hsfunc{$\ldots$}
\end{class}

\renewcommand{\umlfillcolor}{grey}
\begin{class}[text width=9cm]{base}{6,-13.5}
\hstype{Nat : Kind} 
\hstype{Constraint : Kind} 
\hstype{Type : Kind} 
\hstype{Symbol : Kind}
\hstype{CmpSymbol : Symbol -> Symbol -> Ordering} 
\hstype{(\~) : Type -> Type -> Constraint} 
\end{class}

\renewcommand{\umlfillcolor}{grey}
\begin{class}[text width=6cm]{wl-pprint}{-3.4,-6.5}
\hstype{Doc : Type} 
\hstype{instance Show Doc where $\ldots$} 
\hstype{class Pretty a where}
\hstype[20pt]{pretty :: a -> Doc}
\hsfunc{$\ldots$}
\end{class}

\renewcommand{\umlfillcolor}{grey}
\begin{class}[text width=7cm]{simple-sql-parser}{-11,-6.5}
\hstype{QueryExpr : Type where $\ldots$}
\hstype{ValueExpr : Type where $\ldots$}
\hstype{Name : Type where $\ldots$}
\hsfunc{prettyQueryExpr : QueryExpr -> Doc} 
\hsfunc{prettyValueExpr : ValueExpr -> Doc} 
\end{class}


\renewcommand{\umlfillcolor}{applegreen}
\begin{class}[text width=4.9cm]{AmpersandCore}{4,-10}
\hstype{ECArule : Type where $\ldots$} 
%% \hsfunc{$\ldots$}
\end{class}

\renewcommand{\umlfillcolor}{applegreen}
\begin{class}[text width=3cm]{Test}{-2,5}
\hsfunc{$\ldots$}
\end{class}

\renewcommand{\umlfillcolor}{applegreen}
\begin{class}[text width=3cm]{Prototype}{2,5}
\hsfunc{$\ldots$}
\end{class}

\renewcommand{\umlfillcolor}{applegreen}
\begin{class}[text width=4.5cm]{FSpec}{10.4,-10}
\hstype{FSpec : Type where $\ldots$} 
%% \hsfunc{$\ldots$}
\end{class}

\unidirectionalAssociation{Prototype}{}{}{ECA2SQL}
\unidirectionalAssociation{Test}{}{}{ECA2SQL}

\draw [umlcd style, ->] ($(ECA2SQL.south)!0.5!(ECA2SQL.south 
west)$) -- (TypedSQL.north); 
\draw [umlcd style, ->] ($(ECA2SQL.south)!0.5!(ECA2SQL.south 
east)$) -- (TypedSQLCombinators.north); 
\draw [umlcd style, ->] (ECA2SQL.south) -- (Equality.north east); 
\draw [umlcd style, ->] (ECA2SQL.south) -- (Singletons.north west); 
%% \draw [umlcd style, ->] (ECA2SQL.south) -- (Trace.north east); 
\draw [umlcd style, ->] (ECA2SQL.south) -- (Proof Utils.north 
west); 
\draw [umlcd style, ->] (ECA2SQL.south west) -- 
(PrettyPrinterSQL.north); 


\node[above = 1cm of base] (basedummy)       {};
\node[above = 1.5cm of base](basedummy2)       {};
\draw [umlcd style, ->, thick] (basedummy) -- (base); 
\draw [umlcd style dashed line, -, thick] (basedummy2) -- (base); 

\draw [umlcd style, ->] (Singletons.south) -- (Proof Utils.north); 
%% \draw [umlcd style, ->] (Singletons.west) -- (Trace.east); 
\draw [umlcd style, ->] (TypedSQL.south) -- (Equality.north); 
%% \draw [umlcd style, ->] (Equality.south) -- (Trace.north); 
\draw [umlcd style, ->] (TypedSQLCombinators.west) -- 
(TypedSQL.east); 
\draw [umlcd style, ->] (Singletons.west) -- (Equality.east); 
%% \draw [umlcd style, ->] (Proof Utils.south west) -- (Trace.south 
%%east); 
\draw [umlcd style, ->] (TypedSQL.south east) -- 
(AmpersandCore.north west); 

\draw [umlcd style, ->] ([xshift=60pt]TypedSQLCombinators.south) -- 
([xshift=30pt]AmpersandCore.north); 
\draw [umlcd style, ->] (TypedSQLCombinators.south) -- 
(Singletons.north); 
\draw [umlcd style, ->] ([xshift=30pt]TypedSQLCombinators.south) -- 
(Proof Utils.north east); 


\draw [umlcd style, ->] ([xshift=-30pt]PrettyPrinterSQL.south east) 
-- (wl-pprint.north west); 
%% \draw [umlcd style, ->] (PrettyPrinterSQL.south east) -- 
%%(Trace.west); 
\draw [umlcd style, ->] (PrettyPrinterSQL.east) -- (TypedSQL.west); 

%% \node[above = 1cm of Trace] (tracedummy)       {};
%% \node[above = 1.5cm of Trace](tracedummy2)       {};
%% \draw [umlcd style, ->, thick] (tracedummy) -- (Trace); 
%% \draw [umlcd style dashed line, -, thick] (tracedummy2) -- 
%%(Trace); 

\draw [umlcd style, ->] (TypedSQL.south west) -- 
(simple-sql-parser.north east); 

\draw  [umlcd style, ->, fill opacity=0]  
([xshift=70pt]TypedSQLCombinators.south) --++ (0cm,-8cm) -| 
(simple-sql-parser.south);


\draw[umlcd style ,->, fill opacity=0] (ECA2SQL.east) -| node[above 
, sloped , black]{} (FSpec.north);

\draw[umlcd style ,->, fill opacity=0] (ECA2SQL.south) |- 
node[above , sloped , black]{} (AmpersandCore.west);

\unidirectionalAssociation{PrettyPrinterSQL}{}{}{simple-sql-parser}

\end{tikzpicture}
}}\caption{Module diagram for EFA as a whole} \label{fig:efaMod}
\end{figure}


\FloatBarrier

\begin{figure}[!ht]
\makebox[\textwidth][c]{
\scalebox{0.6}{
\begin{tikzpicture}

\begin{package}{TypedSQL}

\begin{class}[text width=30cm]{TypedSQLStatement}{10,18}
\hstype{SQLMethod : [SQLType] -> SQLType -> Type where}
\hstypectr{MkSQLMethod : (ts : [SQLType]) (o : SQLType) -> (Prod 
(SQLValSem . SQLRef) ts -> SQLMthd o) -> SQLMethod ts o}
%% \hstype[20pt]{-> (Prod (SQLValSem . SQLRef) ts -> SQLMthd o) -> 
%%SQLMethod ts o}

\hstype{SQLSem : Kind where}
\hstypectr{Stmt, Mthd : SQLSem}

\hstype{SQLStatement : SQLRefType -> Type = SQLSt Stmt}
\hstype{SQLMthd : SQLRefType -> Type = SQLSt Mthd}

\hstype{SQLSt : SQLSem -> SQLRefType -> Type where} 
\hstypectr{Insert : TableSpec ts -> SQLVal (SQLRel (SQLRow ts)) -> 
SQLStatement SQLUnit}
\hstypectr{Delete : TableSpec ts -> (SQLVal (SQLRow ts) -> SQLVal 
SQLBool) -> SQLStatement SQLUnit}
\hstypectr{Update : TableSpec ts -> (SQLVal (SQLRow ts) -> SQLVal 
SQLBool) -> (SQLVal (SQLRow ts) -> SQLVal (SQLRow ts)) -> 
SQLStatement SQLUnit}
\hstypectr{SetRef : SQLValRef x -> SQLVal x -> SQLStatement SQLUnit}
\hstypectr{NewRef : (a : SQLType) -> IsScalarType a \~ True -> 
Maybe String -> Maybe (SQLVal a) -> SQLStatement (SQLRef a)}
\hstypectr{MakeTable : SQLRow t -> SQLStatement (SQLRef (SQLRel 
(SQLRow t)))}
\hstypectr{DropTable : TableSpec t -> SQLStatement SQLUnit}
\hstypectr{IfSQL : SQLVal SQLBool -> SQLSt t0 a -> SQLSt t1 b -> 
SQLStatement SQLUnit}
\hstypectr{(:>>=) : SQLStatement a -> (SQLValSem a -> SQLSt x b) -> 
SQLSt x b}
\hstypectr{SQLNoop : SQLStatement SQLUnit}
\hstypectr{SQLRet : SQLVal a -> SQLSt Mthd (Ty a)}
\hstypectr{SQLFunCall : SQLMethodRef ts out -> Prod SQLVal ts -> 
SQLStatement (Ty out)}
\hstypectr{SQLDefunMethod : SQLMethod ts out -> SQLStatement 
(SQLMethod ts out)}
\end{class}


\begin{class}[text width=11cm]{TypedSQLLanguage}{0,5}   
\hstype{SQLSizeVariant : Kind where}
\hstypectr{SQLSmall, SQLMedium, SQLNormal, SQLBig : SQLSizeVariant}

\hstype{SQLSign : Kind where}
\hstypectr{SQLSigned, SQLUnsigned : SQLSign}

\hstype{SQLNumeric : Kind where}
\hstypectr{SQLFloat, SQLDouble : SQLSign -> SQLNumeric}
\hstypectr{SQLInt : SQLSizeVariant -> SQLSign -> SQLNumeric}

\hstype{SQLRecLabel : Kind where}
\hstypectr{(:::) : Symbol -> SQLType -> SQLRecLabel}

\hstype{SQLType : Kind where}
\hstypectr{SQLBool, SQLDate, SQLDateTime, SQLSerial : SQLType}
\hstypectr{SQLNumericTy : SQLNumeric -> SQLType}
\hstypectr{SQLBlob : SQLSign -> SQLType}
\hstypectr{SQLVarChar : Nat -> SQLType}
\hstypectr{SQLRel : SQLType -> SQLType}
\hstypectr{SQLRow : [SQLRecLabel] -> SQLType}
\hstypectr{SQLVec : [SQLType] -> SQLType}

\hstype{SQLRefType : Kind where}
\hstypectr{Ty : SQLType -> SQLRefType}
\hstypectr{SQLRef, SQLUnit : SQLType}
\hstypectr{SQLMethod : [SQLType] -> SQLType -> SQLRefType}

\hstype{instance SingKind SQLType where $\ldots$}
\hstype{instance SingKind SQLRefType where $\ldots$}

\hstype{IsScalarType : SQLType -> Bool where $\ldots$}
\hstype{IsScalarTypes : [SQLType] -> Bool where $\ldots$}

\hsfunc{isScalarType : (x : SQLType) -> IsScalarType x}
\hsfunc{isScalarTypes : (x : [SQLType]) -> IsScalarTypes x}
\end{class}


\begin{class}[text width=17.5cm]{TypedSQLExpr}{16,-0.5}
\hstype{SQLVal : SQLType -> Type where}
\hstypectr{pattern SQLScalarVal : IsScalarType a :\~: True -> 
ValueExpr -> SQLVal a}
\hstypectr{pattern SQLQueryVal  : IsScalarType a :\~: False -> 
QueryExpr -> SQLVal a}

\hsfunc{typeOf : SQLVal a -> a}
\hsfunc{argOfRel : SQLRel a -> a} 

\hstype{SQLValSem : SQLRefType -> Type where}
\hstypectr{Unit : SQLValSem SQLUnit}
\hstypectr{Val : (x : SQLType) -> SQLVal x -> SQLValSem (Ty x)}
\hstypectr{pattern Method : Name -> SQLValSem (SQLMethod args out)}
\hstypectr{pattern Ref : (x : SQLType) -> Name -> SQLValSem (SQLRef 
x)}

\hstype{SQLVal : SQLType -> Type = $\lambda$ x $.$ SQLValSem (Ty x)}
\hstype{SQLValRef : SQLType -> Type = $\lambda$ x $.$ SQLValSem 
(SQLRef x)}

\hsfunc{typeOfSem : f `In` [SQLRef, Ty] -> SQLValSem (f x) -> x}

\hsfunc{colsOf : SQLRow xs -> xs}

\hsfunc{unsafeSQLValFromName : (x : SQLType) -> Name -> SQLVal x}
\hsfunc{unsafeSQLValFromQuery : (xs : [SQLRecLabel]) -> NonEmpty xs}
\hsfunc[10pt]{ -> IsSetRec xs -> SQLVal (SQLRel (SQLRow xs))}
\hsfunc{unsafeRefFromName : (x : SQLType) -> Name -> SQLValRef x}

\hsfunc{deref : SQLValRef x -> SQLVal x}   
\end{class}


\begin{class}[text width=15.6cm]{TypedSQLTable}{17,5}
\hstype{TableSpec : [SQLRecLabel] -> Type where}
\hstypectr{MkTableSpec : SQLValRef (SQLRel (SQLRow t))  -> 
TableSpec t}
\hstypectr{TableAlias : (ns : [Symbol]) -> IsSetRec ns }
\hstype[20pt]{-> TableSpec t -> TableSpec (ZipRec ns (RecAssocs t))}

\hsfunc{typeOfTableSpec : TableSpec t -> SQLRow t}
\hsfunc{typeOfTableSpec : TableSpec t -> t}

\hsfunc{tableSpec : Name -> Prod (K String :*: Id) tys}
\hsfunc[10pt]{ -> exists (ks : [SQLRecLabel]) (Maybe (RecAssocs ks 
:\~: tys, TableSpec ks))}

\end{class}

\draw [umlcd style, ->] ([xshift=-60pt]TypedSQLStatement.south) -- 
([xshift=-60pt]TypedSQLStatement |- TypedSQLExpr.north); 
\draw [umlcd style, ->] ([xshift=60pt]TypedSQLStatement.south) -- 
([xshift=60pt]TypedSQLStatement |- TypedSQLTable.north); 
\draw [umlcd style, ->] ([xshift=-180pt]TypedSQLStatement.south) -- 
([xshift=-180pt]TypedSQLStatement |- TypedSQLLanguage.north); 


\draw [umlcd style, ->] (TypedSQLTable.south) -- (TypedSQLTable |- 
TypedSQLExpr.north); 
\draw [umlcd style, ->] (TypedSQLTable.west) -- (TypedSQLTable -| 
TypedSQLLanguage.east); 
\draw [umlcd style, ->] (TypedSQLExpr.west) -- (TypedSQLExpr -| 
TypedSQLLanguage.east);

\end{package}

\end{tikzpicture}
}}\caption{Module diagram for TypedSQL} \label{fig:typedSQL}
\end{figure}


\subsubsection{TypedSQL}

The module hierarchy for the TypedSQL module is shown in
figure \ref{fig:typedSQL}. The submodules of TypedSQL are quite large, 
however, the majority of the definitions within are type and kind definitions, 
which correspond precisely to entities defined by MySQL. The only exception is 
SQLRel, which distinguishes relations from scalar types - MySQL does not make 
this distinguishment. 
Only a few
helper functions are defined in these modules -- namely, only things which 
form
the core interface to TypedSQL and in particular, SQLStatement. These 
functions
cannot be defined outside of the module, usually because they use an 
abstract
constructor. By making the core interface to TypedSQL very small, 
maintaining 
the TypedSQL language definition separately from the implementation of 
EFA is simplified. All of the data types in TypedSQL are correct by 
construction, with the exception of the functions explicitly labeled 
``unsafe''. These functions (unsafeSQLValFromName, unsafeSQLValFromQuery, and 
unsafeRefFromName)
are required only when implementing a new SQL primitive on top of the SQL
language - they are not intended for regular use. 

The TypedSQLLanguage module models the SQL type language in Haskell with a
series of kind declarations. The language being modeled is only a subset of 
the
SQL type language, corresponding approximately to the subset which the core
Ampersand system already uses. The meaning of each Haskell type corresponds
exactly to the appropriate MySQL type, which are detailed in the MySQL
manual \citep{mySQLman}. Similarly, the constructors of SQLSt all correspond
to different varieties of SQL statements -- for the majority of 
constructors,
there is a one-to-one correspondence between the semantics of the 
constructor,
and the semantics of the SQL statement with the same name. The exceptions 
are:     
\begin{description}
\item[\texttt{SQLNoop}] MySQL does not have a primitive no-op statement
\item[\texttt{SQLDefunMethod}]  MySQL does not allow defining 
procedures within procedures; 
this constructor denotes that a method ``defined'' within another 
statement must 
first be loaded as a MySQL Stored Procedure \citep{mySQLman}.
\item[\texttt{:>>=}] This constructor corresponds to sequencing 
statements. This constructor
embeds scope checking of MySQL statements in the Haskell compiler -- 
ill-formed statements
containing variables which are not defined (i.e. not in scope) will be 
rejected by the Haskell
compiler. 
\end{description}

The SQLSt data type also distinguishes between two varieties of statements:
SQLStatement and SQLMthd. The former is the type of regular statements, 
while the latter is the type of ``almost'' complete methods - methods whose 
formal parameters have not yet been bound. This is done in order to statically
guarantee that a SQL method always returns a value. Due to the type of
\lstinline{:>>=}, this also rules out SQL programs which contain dead code -- no
code can follow SQLRet, which is always guaranteed to return from the 
function.
While this does rule out some valid programs (for example, an if statement 
in
which both branches end with a return, but there is no return following the 
if
statement, will be rejected), these programs can be written in an equivalent
way in our language without any loss of generality.    

\subsubsection{TypedSQLCombinators}

The module TypedSQLCombinators, whose members are given in 
figure~\ref{fig:typedSQLComb},
implements a subset of primitive SQL functions on top of the TypedSQL 
expression type. 
The data type PrimSQLFunction encodes the specification of each 
function; the type
and semantics of each function is that of the corresponding function in 
MySQL (refer to
the MySQL manual ~\cite{mySQLman} for details on each function). The 
only exception
is the \lstinline{Alias} function, which is a primitive syntactic 
constructor (not a named entity)
in MySQL - rows can be aliased with a select statement. Aliasing a row 
means to change
the name of each association in the row, but not the shape of the row 
(i.e. the types of
each element of the row, as well as their ordering). The single 
function \lstinline{primSQL}
implements all of the primitive SQL functions. It takes as an argument 
a specification
of the primitive function, a tuple of arguments of the correspond 
types, and returns
a SQL value, again of the corresponding type. 

\begin{figure}[!ht]
    \makebox[\textwidth][c]{
        \scalebox{0.6}{
            \begin{tikzpicture}
            
            %% \begin{package}{TypedSQL}
            \begin{class}[text width=22cm]{TypedSQLCombinators}{17,5}
            \hstype{PrimSQLFunction : [SQLType] -> SQLType -> Type}
            \hstypectr{PTrue, PFalse : PrimSQLFunction [] SQLBool}
            \hstypectr{Not : PrimSQLFunction [ SQLBool ] SQLBool}
            \hstypectr{Or, And : PrimSQLFunction [ SQLBool, SQLBool ] 
                SQLBool}
            \hstypectr{In, NotIn : PrimSQLFunction [ a, SQLRel a ] SQLBool}
            \hstypectr{Exists : PrimSQLFunction [ SQLRel a ] SQLBool}
            \hstypectr{GroupBy : PrimSQLFunction [ SQLRel a, a ] (SQLRel 
                (SQLRel a))}
            \hstypectr{SortBy  : PrimSQLFunction [ SQLRel a, a ] (SQLRel a)}
            \hstypectr{Max, Min, Sum, Avg : IsSQLNumeric a -> 
                PrimSQLFunction [ SQLRel a ] a }
            \hstypectr{Alias : (RecAssocs ts :~: RecAssocs ts') -> 
                PrimSQLFunction [ SQLRel (SQLRow ts) ] (SQLRel (SQLRow ts'))}
            
            \hsfunc{primSQL : forall (args : [SQLType]) (out : SQLType) -> 
                PrimSQLFunction args out -> Prod SQLVal args -> SQLVal out}
            
            \hsfunc{(!) : (xs : [SQLType]) (i : Symbol) -> LookupRec xs i 
                :\~: r -> SQLVal (SQLRow xs) -> SingT i -> SQLVal r}
            \hsfunc{(!) : (xs : [SQLType]) (i : Symbol) -> LookupRec xs i 
                :\~: r -> SQLVal (SQLRel (SQLRow xs)) -> SingT i -> SQLVal r}
            \hsfunc{(!) : (xs : [SQLType]) (i : Nat) -> LookupIx t i :\~: r 
                -> SQLVal (SQLVec t) -> SingT i -> SQLVal r}
            
            
            \end{class}
            
            %% \end{package}
            
            \end{tikzpicture}
        }}\caption{Module diagram for TypedSQLCombinators} 
        \label{fig:typedSQLComb}
\end{figure}





\begin{figure}[!ht]
\makebox[\textwidth][c]{
    \scalebox{0.6}{
        \begin{tikzpicture}
        
        \begin{class}[text width=22cm]{Equality}{17,5}
        
        \hsfunc{cong : f :\~: g -> a :\~: b -> f a :\~: g b}
        \hsfunc{cong2 : f :\~: g -> a :\~: a' -> b :\~: b' 
        -> f a b :\~: g a' b'}
        \hsfunc{cong3 : f :\~: g -> a :\~: a' -> b :\~: b' 
        -> c :\~: c' -> f a b c :\~: g a' b' c'}
        
        \hstype{Dict : Constraint -> Type}
        \hstypectr{Dict : p => Dict p}
        
        \hstype{Exists : (k -> Type) -> Type} 
        \hstypectr{Ex : p x -> Exists p}
        
        \hsfunc{($\#$>>) :: Exists p -> (forall x -> p x -> 
        r) -> r}
        
        \hstype{Not : Type -> Type} 
        
        \hstype{class NFData a}
        
        \hstype{doubleneg : NFData a => a -> Not (Not a)} 
        
        %% Why is this particular defintion off by about 
        %%2pt? Very strange...
        \hstype[-2pt]{triviallyTrue : Not (Not ())}
        
        \hstype{mapNeg : (NFData a, NFData b) => (b -> a) 
        -> Not a -> Not b}
        
        \hstype{elimNeg : NFData a => Not a -> a -> Void}
        
        \hstype{data Void where}
        
        \hstype{data Dec : Type -> Type}
        \hstypectr{Yes : !p -> Dec p}
        \hstypectr{No  : !(Not p) -> Dec p}
        
        \hstype{DecEq : k -> k -> Type = $\lambda$ a b$.$ 
        Dec (a :\~: b)}
        
        \hsfunc{mapDec : (p -> q) -> (Not p -> Not q) -> 
        Dec p -> Dec q}
        
        \hsfunc{liftDec2 :: Dec p -> Dec q -> (p -> q -> r) 
        -> (Not p -> Not r) -> (Not q -> Not r) -> Dec r}
        
        \hsfunc{dec2bool :: DecEq a b -> Bool}
        
        \end{class}
        
        \end{tikzpicture}
    }}\caption{Module diagram for Equality} 
    \label{fig:equality}
\end{figure}

\subsubsection{Equality}

The Equality module (\ref{fig:equality}) defines several 
utilities for working
with proof-like values, including the existential 
quantification data type,
proofs of congruence of propositional equality of various 
arities -- \lstinline{cong, cong2, cong3} -- and the Dec type, which 
encodes the concept of
a decidable proposition. The most important element of this 
module, however is
the abstract \lstinline{Not} type. We must prove certain 
things about our
program to the Haskell type system. For example, if one 
attempts to construct a
scalar SQLVal for some SQLType $S$, one must first prove 
that that type is a
scalar type. The main use of this is decidable equality, 
which is similar to
regular equality, but additionally to giving a ``yes'' or 
``no'' answer, it also
stores a \emph{proof} of that answer.

The view of propositions as types comes from the 
Curry-Howard
isomorphism \cite{props}; however, this is not quite sound 
in Haskell, because
every type is inhabited by $\bot$, which corresponds to 
\texttt{undefined}, an
exception, or non-termination. Due to laziness, an 
unevaluated $\bot$ can be
silently ignored. At worst, this corresponds to a sound use 
of
\lstinline{unsafeCoerce} leaking into the ``outside 
world'', that is, allowing
a user to accidently expose the use of an 
\lstinline{unsafeCoerce}. One can
usually work around this by evaluating all proof-like 
values to normal form
before working with them (this is accomplished by the 
\lstinline{NFData} class,
which stands for normal form data). The normal form of most 
datatypes contains
precisely one ``type'' of bottom -- namely the value 
$\bot$, as opposed to
$\bot$ wrapped in a constructor, for example 
\lstinline{Just $\bot$}. This
bottom can then be removed with the Haskell primitive 
\lstinline{seq},
producing a value which can soundly be used as a proof.

A problem arises when we consider the negation of a proposition. $\lnot p$ 
is typically encoded in Haskell as \lstinline{p -> $\bot$}, where the type 
$\bot$
can be represented by any uninhabited type, typically called\lstinline{Void}. 
However, the normal form of a function can contain any number of $\bot$ hidden 
deep inside the function, because evaluating a function to normal form only 
evaluates up to the outermost binder.  To prevent any
unsoundness which this might cause, values of type \lstinline{Not p} are reduced
to normal form as they are built, starting with a canonical value which is known
to be in normal form - the value \lstinline{triviallyTrue}.  This is the role of
NFData in the type signature of \lstinline{mapNeg}.  As mentioned previously,
the type \lstinline{Not} is abstract, so the provided interface, which is known
to be sound, is the only way to construct and eliminate values of type
\lstinline{Not p}.


\subsubsection{PrettySQL}

The module PrettySQL defines pretty printers for each of the types
corresponding to SQL entities, including SQL types, SQL values, SQL references
and methods, and SQL statements. These pretty printers produce a value of
type Doc (which comes from the wl-pprint package), which is like a string
, but contains the layout and indentation of all lines of the document,
allowing for easy composition of Doc values into larger documents, without
worrying about layout. The SQL entities are pretty printed in a human
readable format, complete with SQL comments which indicate the origin
and motive of generated code. 


\begin{figure}[!ht]
    \makebox[\textwidth][c]{
        \scalebox{0.6}{
            \begin{tikzpicture}
            
            \begin{class}[text width=22cm]{PrettySQL}{17,5}
            
            \hstype{instance Pretty (SQLTypeS x) $\,\,$ where 
                $\ldots$}
            \hstype{instance Pretty (SQLVal x)  $\,\,$ where 
                $\ldots$}
            \hstype{instance Pretty (SQLValSem x)  $\,\,$ where 
                $\ldots$}
            \hstype{instance Pretty (SQLSt k x)  $\,\,$ where 
                $\ldots$}
            \hstype{instance (str \~ String) => Pretty (str, 
                SQLMethod args out)  $\,\,$ where $\ldots$}
            
            \end{class}
            
            \end{tikzpicture}
        }}\caption{Module diagram for PrettySQL} 
        \label{fig:prettySQL}
    \end{figure}

\subsubsection{Proof Utils}

The ProofUtils modules, shown in 
figure~\ref{fig:utilsMod} provides various
utilities for proving compile time invariants, 
including the definitions of
various predicates used in other modules, as well as 
the value-level functions
which prove or disprove those predicates. Generally 
speaking, a predicate is a
type level function which returns either a true or 
false value, or has kind
\lstinline{Constraint}, in which case truth corresponds 
to a satisfied
constraint, while false to an unsatisfied one. Many 
predicates have value level
witnesses as well; these are datatypes which are 
inhabited if and only if the
predicate is true.  Therefore, pattern matching on the 
predicate witness can be used to recover a proof of the predicate at run time.

The function of the most important predicates and types is briefly summarized 
as follows. 

\begin{figure}[!ht]
\makebox[\textwidth][c]{
    \scalebox{0.6}{
        \begin{tikzpicture}
        
        \begin{class}[text width=22cm]{ProofUtils}{17,5}
        
        \hstype{Prod : (f : k -> Type) -> (xs : [k]) -> 
            Type} 
        \hstypectr{PNil : Prod f []}
        \hstypectr{PCons : f x -> Prod f xs -> Prod f 
            (x : xs)}
        
        \hstype{Sum : (f : k -> Type) -> (xs : [k]) -> 
            Type}  
        \hstypectr{SHere : f x -> Sum f (x : xs)}
        \hstypectr{SThere : Sum f xs -> Sum f (x : xs)}
        
        
        \hstype{class All (c : k -> Constraint) (xs : 
            [k]) where}
        \hstypectr{mkProdC : Proxy c -> (forall a -> c 
            a => p a) -> Prod p xs}
        
        \hstype{instance All (c : k -> Constraint) [] 
            where $\ldots$}
        \hstype{instance (All c xs, c x) => All c (x : 
            xs) where $\ldots$}
        
        \hstype{Elem : k -> [k] -> Type where}
        \hstypectr{MkElem : Sum (:\~: x) xs -> Elem x 
            xs} 
        
        \hstype{IsElem : (x : k) -> (xs : [k]) -> 
            Constraint where $\ldots$} 
        \hstype{AppliedTo : (x : k) -> (f : k -> Type) 
            -> Type}
        \hstypectr{Ap : f x -> x `AppliedTo` f}
        
        \hstype{(:.:) : (f : k1 -> Type) -> (g : k0 -> 
            k1) -> (x : k0) -> Type}
        \hstypectr{Cmp : f (g x) -> (:.:) f g x} 
        
        \hstype{(:*:) : (f : k0 -> Type) -> (g : k0 -> 
            Type) -> (x : k0) -> Type} 
        \hstypectr{(:*:) : f x -> g x -> (:*:) f g x} 
        \hstype{K : (a : Type) -> (x : k) -> Type} 
        \hstypectr{K : a -> K a x} 
        \hstype{Id : (a : Type) -> Id a}
        \hstypectr{Id : a -> Id a} 
        
        \hstype{Flip : (f : k0 -> k1 -> Type) -> (x : 
            k1) -> (y : k0) -> Type} 
        \hstypectr{Flp : f y x -> Flip f x y} 
        
        \hstype{(&&) : (x : Bool) -> (y : Bool) -> Bool 
            where $\ldots$} 
        
        \operation{\lstinline[mathescape]{(|&&) : (a : 
                Bool) -> (b : Bool) -> a && b}}
        
        \hstype{RecAssocs : [RecLabel a b] -> [b] where 
            $\ldots$} 
        \hstype{RecLabels : [RecLabel a b] -> [b] where 
            $\ldots$} 
        
        \hstype{IsSetRec\_ : [RecLabel a b] -> [a] -> 
            Constraint where $\ldots$} 
        
        \hstype{IsSetRec : [RecLabel a b] -> Constraint 
            where $\ldots$} 
        \hstype{SetRec\_ : [a] -> [RecLabel a b] -> 
            Type} 
        \hstype{SetRec : [RecLabel a b] -> Type = 
            SetRec\_ []} 
        
        
        \hsfunc{openSetRec : forall (xs : [RecLabel k 
            k']) $\,\,$r0 -> SingKind k => SetRec xs -> 
            (IsSetRec xs => r0) -> r0}
        
        \hsfunc{decNotElem : forall (xs : [a]) $\,\,$x 
            -> SingKind a => xs -> x -> Dec (NotElem xs x)}
        
        \hsfunc{decSetRec : forall (xs : [RecLabel a 
            b]) -> SingKind a => xs -> Dec (SetRec xs)}
        
        \hstype{LookupRecM : [RecLabel Symbol k] -> 
            Symbol -> Maybe k where $\ldots$} 
        
        \hsfunc{lookupRecM : forall (xs : [RecLabel 
            Symbol k]) (x : Symbol) -> xs -> x -> 
            LookupRecM xs x}
        
        \hstype{ZipRec : [a] -> [b] -> [RecLabel a b] 
            where $\ldots$} 
        
        \hstype{IsNotElem :  [k] -> k -> Constraint 
            where $\ldots$} 
        \hstype{NotElem : [k] -> k -> Type}
        
        \hstype{And : (xs : [Bool]) -> Bool where 
            $\ldots$} 
        
        \hsfunc{and\_t : (xs : [Bool]) -> And xs}
        
        \hsfunc{compareSymbol : (x : Symbol) -> (y : 
            Symbol) -> CmpSymbol x y}
        
        
        \hstype{NonEmpty : (xs : [k]) -> Constraint 
            where} 
        \hstypectr{NonEmpty (x : xs) = ()} 
        
        \hsfunc{prod2sing : forall (xs : [k]) -> 
            SingKind k => Prod SingT xs -> SingT xs} 
        \hsfunc{sing2prod : forall (xs : [k]) -> 
            SingKind k => SingT xs -> Prod SingT xs} 
        \hsfunc{foldrProd : forall acc (f : k -> 
            Type)$\,\,$xs -> acc -> (forall q -> f q -> acc 
            -> acc) -> Prod f xs -> acc}
        \hsfunc{foldlProd : forall acc (f : k -> 
            Type)$\,\,$xs -> acc -> (forall q -> f q -> acc 
            -> acc) -> Prod f xs -> acc}
        \hsfunc{mapProd : forall (f : k -> Type)$\,\,$g 
            -> (forall x -> f x -> g x) -> Prod f xs -> 
            Prod g xs}
        \hsfunc{foldrProd' : forall (f : k -> 
            Type)$\,\,$xs1 -> (forall x xs -> f x -> Prod g 
            xs -> Prod g (x : xs)) -> Prod f xs1 -> Prod g 
            xs1 }
        \hsfunc{someProd : [Exists f] -> Exists (Prod 
            f)}
        
        \end{class}
        
        \end{tikzpicture}
    }}\caption{Module diagram for ProofUtils} 
    \label{fig:utilsMod}
\end{figure}

\begin{description}
    \item[\texttt{Prod}] The type \lstinline{Prod f xs} 
    represents the $n$-ary product of the type
    level list $xs$, with each $x \in xs$ being mapped 
    to the type \lstinline{f x}. This 
    type is accompanied by the functions 
    \lstinline{prod2sing, sing2prod} which convert
    between singletons and products; the functions 
    \lstinline{foldrProd, foldlProd, foldrProd'}, 
    which are all eliminators for \lstinline{Prod} 
    (several eliminators are needed because the 
    most general eliminator is not well typed in 
    Haskell). 
    \item[\texttt{Sum}] The same as above, but the 
    $n$-ary sum as opposed to product. 
    \item[\texttt{All}] The constraint 
    \lstinline{All c xs} holds if and only if 
    \lstinline{c x} holds for 
    all $x \in xs$. 
    \item[\texttt{Elem, IsElem}]
    \lstinline{IsElem x xs} holds precisely when $x \in 
    xs$. 
    \lstinline{Elem} is the witness of 
    \lstinline{IsElem}. 
    \item[\texttt{AppliedTo, Ap, :.:, Cmp, :*:, K, Id, 
    Flip}] Categorical data types which encode
    a generalized view of algebraic data types. This 
    approach is largely standardized (and is only
    replicated here to avoid incurring a large 
    dependency) -- for more information, 
    see~\cite{alacarte}. 
    \item[\texttt{RecAssocs, RecLabels, ZipRec}] Type 
    level functions for working with type level 
    records. 
    A record in this context is a list of types of some 
    kind, each associated with a unique string label. 
    \lstinline{RecAssocs, RecLabels} retrieve the 
    associations and labels of a record, respectively, 
    while \lstinline{ZipRec} constructs such a record 
    from the associations and labels. It is the case
    that 
    \lstinline{ZipRec (RecAssocs x) (RecLabels x) == x} for all 
    \lstinline{x} .
    \item[\texttt{IsSetRec, SetRec}] The predicate 
    \lstinline{IsSetRec x} holds precisely if 
    \lstinline{x}
    is a valid record type, whose labels are all 
    unique. \lstinline{SetRec} is the witness for 
    \lstinline{IsSetRec}. 
    \item[\texttt{IsNotElem, NotElem}] The predicate 
    \lstinline{IsNotElem x xs} holds precisely if 
    $\lnot x \in xs$.
    \lstinline{NotElem} is the witnss for 
    \lstinline{IsNotElem}. 
    \item[\texttt{\&\&,And}] Binary and $n$-ary boolen 
    conjunction, with the usual semantics. 
\end{description}
    
    
\subsubsection{Singletons}\label{subsec:Singletons}

The Singletons module (figure ~\ref{fig:singletons}) is 
not fully detailed here; rather, a vastly simplified version is presented. The
\lstinline{SingT} type denotes a generic singleton for 
any kind which implements
\lstinline{SingKind} -- then, the main operation of 
interest on singletons is
decidable equality, which is realized by the function 
\lstinline{%==}. The
    detailed implementation is omitted because the 
    singletons approach in Haskell
    is well known and well 
    documented~\cite{singletons}. We re-implement
    singletons instead of using the well established 
    \texttt{singletons} library
    because, while this library is very well written, 
    it relies very heavily on
    Template Haskell~\cite{th}, which is essentially 
    string-based
    metaprogramming. Template Haskell is extremely 
    error prone and very difficult
    to maintain. As one of our primary goals is 
    long-term maintainability, and
    Template Haskell changes, sometimes drastically, 
    with every new release of
    GHC, including it in this project was deemed not 
    worth the
    headache. Therefore, we have reimplemented 
    singletons without Template
    Haskell, at the cost of having to write slightly 
    more boilerplate.


\begin{figure}[!ht]
    \makebox[\textwidth][c]{
        \scalebox{0.6}{
            \begin{tikzpicture}
            
            \begin{class}[text width=22cm]{Singletons}{17,5}
            \hstype{SingT : k -> Type} 
            \hstype{class SingKind (k : Kind) where $\ldots$} 
            \hsfunc{(\%==) : forall (x : k) (y : k) -> SingKind k => x 
                -> y -> DecEq x y}
            
            \end{class}
            
            \end{tikzpicture}
        }}\caption{Module diagram for Singletons} \label{fig:singletons}
\end{figure}

\subsection{Key Algorithm}
The key algorithm for the EFA project is AMMBR \cite{AMMBR}. AMMBR is a method 
that allow organizations to build information systems that comply to their 
business requirements in a provable manner. This algorithm is implemented in 
Ampersand and is responsible for translating the business requirements into ECA 
rules. These ECA rules contain information on how to fix any data violation and 
are translated into SQL queries in our EFA project.

\subsection{Communication Protocol}
The EFA implementation needs to communicate with the front end to be able to 
run the generated SQL queries when a violation occurs. 
\begin{itemize}
    \item \textbf{Old communication protocol -  PHP engine} \\
    In the existing version, Ampersand depends on PHP code to run the 
    generated SQL on the database. However, this comes at the cost of 
    human intervention, which results in manual maintenance when 
    changes occur during development. 
    \item \textbf{New Communication protocol - Stored Procedures} \\
    The developments teams of EFA has come to a conclusion that the 
    best way of communicating with the front-end will be to use Stored 
    Procedures\cite{SP}. These Stored Procedures provide the extra 
    benefit of query optimization at compile time which results in 
    better performance. While this is a suggested change, it will 
    require changes to the existing Ampersand software in order for 
    this idea to be successfully implemented. This anticipated 
    change will be implemented in the near future.
    
\end{itemize}
