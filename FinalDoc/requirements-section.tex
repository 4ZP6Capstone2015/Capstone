\chapter{Functional Requirements}\label{sec:Functional}

\edcomm{WK}{Intro missing.}

\section{System Requirements}
%%-----------------------------SIDE EFFECTS---------------------------------%%
{\setlength{\tabcolsep}{6pt} %% Default is 6
    \begin{tabularx}{\textwidth}{>{\bfseries}m{3cm}X}
        Requirement & S1 \\ 
        \midrule
        \endhead
        Description  & Create pure functions with no unintended side effects
        \\	Rationale & The use of a functional programing languages requires 
        that this program be a pure function and does not have side effects, 
        however certain portions of the code requires the execution of side 
        effects to match the behaviour presented by external programs. In these 
        specific instances, the side effects are an intended behaviour.
        \\	Originator & Stakeholder/Developer
        
        \\ Test Case & Desired results can be confirmed as they will be 
        reflected in changes that take place in the Ampersand database.
        \\	Customer Satisfaction & 5 - Highest 
        \\	Priority & 5 - Highest 
        \vspace{12pt}
    \end{tabularx}
}

%%------------------ MODULES MUST FIT AMPERSAND FRAMEWORK-----------------%%

{\setlength{\tabcolsep}{6pt} %% Default is 6
    \begin{tabularx}{\textwidth}{>{\bfseries}m{3cm}X}
        Requirement & S2 \\ 
        \midrule
        \endhead
        Description  & Added modules must fit within Ampersand's current 
        framework
        \\	Rationale & Ampersand is a huge system that has weekly additions 
        to prevent conflict and breaking of existing packages/modules, an 
        effort should be made to minimize external dependencies. As EFA will be 
        an internal component of Ampersand, if a package that EFA depends on to 
        function properly is no longer maintained and breaks, it will in turn 
        break Ampersand.
        \\	Originator & Ampersand Creators (i.e. our client)        

        \\ Test case & Added modules are tested with cabal build inside of the
        Ampersand system as an internal component (i.e. System testing)
        \\	Customer Satisfaction & 4 - High 
        \\	Priority & 4 - High
        \vspace{12pt}
    \end{tabularx}
}
{\setlength{\tabcolsep}{6pt} %% Default is 6
    \begin{tabularx}{\textwidth}{>{\bfseries}m{3cm}X}
        Requirement & S3 \\ 
        \midrule
        \endhead
        Description  & All code must be maintainable. 
        \\	Rationale & For a system such as Ampersand to be maintainable, all 
        code for each of its components must be well documented so it may be 
        easily understood by those that were not a part of its original 
        development.
        \\	Originator & Ampersand Creators (i.e. our client)        
        
        \\ Test case & A literate program that produces a
        \edchange{WK}{latex}{\LaTeX{}} document. \edcomm{WK}{I do hope
          you still strive to pass this by adding the literate
          documents to the appendix!}
        \\	Customer Satisfaction & 4 - High 
        \\	Priority & 4 - High
        \vspace{12pt}
    \end{tabularx}
}

\section{Project Requirements}
%%-------------------------TYPE CORRECTNESS ----------------------------------%%
{\setlength{\tabcolsep}{6pt} %% Default is 6
    \begin{tabularx}{\textwidth}{>{\bfseries}C{3cm}X}
        Requirement & P1 \\ 
        \midrule
        \endhead
        Description  & Provable Correctness: \edcomm{WK}{The use of
          dependent types in TypedSQL establishes some (safety)
          properties, but not the correctness proofs for the
          ECA-to-SQL conversion (see P2) that the Ampersand developers would be
          interested in. In its current form, this ``requirement''
          therefore looks quite confusing/confused.} Haskell like other functional 
        programming languages have 
        a strong type system which can be used for machine-checked proofs.
        \\	Rationale & Curry-Howard correspondence which states that the 
        return type of the function is analogous to a logical theorem, that is 
        subject to the hypothesis corresponding to the types of the argument 
        values that are passed to the function and thus the program used to 
        compute that function is analogous to a proof of that theorem.
        \\	Originator & Ampersand Creators
        \\ Test Cases & Using QuickCheck to test function properties.
        \\	Priority & 4 - High
        \vspace{12pt}
    \end{tabularx}
}
%%--------------------------CORRECTNESS OF ECA TO SQL RULES ------------------%%
{\setlength{\tabcolsep}{6pt} %% Default is 6
    \begin{tabularx}{\textwidth}{>{\bfseries}C{3cm}X}
        Requirement & P2 \edcomm{WK}{This is the main requirement. Why so late?}\\ 
        \midrule
        \endhead
        Description  & Generated SQL queries must preserve the semantics of ECA 
        rules.  
        \\	Rationale & The translation would otherwise not be correct, as the 
        rules would be meaningless if their semantics are lost.
        \\	Originator & Ampersand Creators
        \\ Test Cases & Internal structure of ECA rules can be compared to SQL 
        queries through a series of datatype tests, each of which will result 
        in a traceable result or error message
        \\	Priority & 4 - High
        \vspace{12pt}
    \end{tabularx}
}
{\setlength{\tabcolsep}{6pt} %% Default is 6
    \begin{tabularx}{\textwidth}{>{\bfseries}C{3cm}X}
        Requirement & P3 \\ 
        \midrule
        \endhead
        Description  & Generated SQL queries must be correctly implemented.
        \edcomm{WK}{What do you mean by this? Would perhaps the
          syntax- and type-correctness of the SQL statements coming
          out of TypedSQL, and therewith the rationale of P1, be
          appropriate here?}
        \\	Rationale & Ampersand uses a MySQL database, for queries to be 
        recognized and executed they must be error free. 
        \\	Originator & Ampersand Creators
        \\ Test Cases & Using MySQL WorkBench queries are manually executed and 
        checked for errors.
        \\	Priority & 4 - High
        \vspace{12pt}
    \end{tabularx}
}
