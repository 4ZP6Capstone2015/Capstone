%% \chapter*{Appendix: Literate Haskell Code}


\chapter*{Appendix: Notes on Implementation Issues}

This section contains a detailed list of concerns that came to light during the
course of software implementation. Though some issues were addressed, others are
left unresolved as there is no known solution for them. This list is meant to
provide an overview of the various problems -- and their solutions, if
applicable -- which were encountered during the development of EFA. 

\begin{description}[leftmargin=1em]
\item[No kind equality] 
  One would naturally like to implement a singleton type as
  \begin{lstlisting}
  data SingT (a::k) where
    SingApp :: SingT f -> SingT a -> SingT (f a) 
  \end{lstlisting}
  in order to represent the singleton of the application of a type
  constructor.  However, such a type, while possible to define, is essentially
  impossible to use, because the typechecker (as of GHC 7.10) does not know
  about the equality of kinds. Namely, when the typechecker sees an equality
  between \emph{types}, it knows that the left-hand-side and right-hand-side
  are interchangeable in all contexts, but it does not derive the same
  information from an equality on \emph{kinds}.
  
  This feature is currently planned to be made available in GHC in version 8.0
  (the next stable release) in the form of the the \texttt{TypeInType}
  extensions~\cite{typeintype}.

\item[Non-injective type families] 
  Specifically, lack of injectivity in type functions means that GHC cannot infer
  an instantiation of a type variable that appears only under type
  families. Recently, Microsoft researchers have implemented a GHC modification
  that allows type functions to be annotated as injective and plan to make it
  available with the next stable release of the compiler \cite{microsoft}.

\item[Functional dependencies are \emph{not} equivalent to type families] 
  For example, given:
  \begin{lstlisting}
  class $\ldots$ => $\,\,$ C a b | a -> b
  \end{lstlisting}
  the typechecker knows that if the type $a$ is known and there is a constraint
  \lstinline{C a b} given, then the type $b$ is known as well. In other words, the functional 
  dependency $\ldots$\lstinline{| a -> b} asserts that the class \lstinline{C} is
  actually a function. \\ 
  However, you cannot write
  \begin{lstlisting}
  injec :: (C a b, C a b') => b :~: b' 
  \end{lstlisting}
  as one would expect -- this property is clearly true of a true math function --
  because of global incoherence caused by module boundaries. No solution 
  is currently known \cite{sof}.

%% \eddelete{YT}{This should not be here? Also doesn't make one bit of sense}{
%% \item \textbf{Types as propositions.} 
%% This is assuming that one exists; 
%% simply put, it describes a correspondence between a given logic and a given 
%% programming language, where we have "simplification of proofs as evaluation 
%% of programs" \cite{props}.
%% }

\item[Exceeding complex type level invariants] Certain invariants are difficult to state and prove,
  for example, the type of the \texttt{SELECT} SQL operator for the \texttt{TypedSQL}
  query language. The type for this function in particular was too difficult to even 
  write down, let alone write an implementation which satisfied the invariants 
  in the type. The solution to this problem is to use a vastly simplified term language,
  without losing the generality of expressing certain interesting expressions. 

\item[Singletons have a lot of boilerplate code] The \texttt{singletons}
  package solves this problem by using Template Haskell to generate the
  boilerplate. However, TH is little more than string-based metaprogramming,
  so it is very clumsy to use, but more importantly nearly unmaintainable,
  as the internals of the Template Haskell library change with nearly every
  compiler version, and when code generated by TH breaks, it is very difficult
  to debug. 

  Our reimplemenation of the singleton pattern simply accepts the fate of 
  having to write some boilerplate. However, the design of our implemenation 
  ensures that for every type constructor, one only needs to write a constant 
  amount of boilerplate, and then one can implement functions generically (for 
  all singleton kinds) in terms of those functions. This is in constrast 
  to the \texttt{singletons} library, which would require more boilerplate 
  for every additional feature to be implemented on singleton types. 

\item[Defining a typed term language from an untyped one]
  The TypedSQL query language encodes at the type level the type of the 
  value which is represented by any given expression. However, this
  type is internally implemented as a wrapper around an abstract syntax
  tree which does \emph{not} encode the type of the expression as a Haskell
  type. Users of the TypedSQL code must not be allowed to create typed
  expresions from untyped ones erroneously, that is, by assigning an incorrect
  type to the untyped expression. 
  
  The solution is abstraction barriers restricting access to the implementation of
  the TypedSQL query language. The approved way of constructing such expressions
  is through the use of the \emph{typed} query functions provided by
  \texttt{TSQLCombinators}. However, in order to inject untyped expressions
  into typed ones when one is truly certain that the assigned type is correct,
  functions marked ``unsafe'' are still exported for the construction of 
  typed expressions from untyped ones. 

\item[Relationship between SQL schemata and relation algebra]
  \edcomm{WK}{This is part of what I would have expected in ``Key Algorithm''.}
  Ampersand represents the underlying information system in terms of relation algebra
  expressions representing entities in the information system and relations between them.
  Ampersand maps relation algebra entities to SQL schemas in a such a way that 
  there is minimal duplication of data. Therefore, the source and target of some 
  relation may actually be columns of some larger table 

  For example the relations 
  \begin{verbatim} 
    r0 : A -> B [ FUN, INJ ] 
    r1 : A -> C [ FUN, INJ ]  
  \end{verbatim}
  \vspace{-2em}
  can be modelled as a single table with three columns -- $A, B$ and $C$ --
  where the first column contains an element $a$ iff that element is a memeber
  of \texttt{I[A]}, the second and third columns contain a non-null value $b$ or
  $c$ iff $a\,\,r0\,\,b$ or $a\,\,r0\,\,c$ respectively. 

 Since the schemas of tables are encoded at the type level, these complex relations
  must also be computed at the type level.

  The previously mentioned technique of dealing with exceeding complex type level invariants
  is also applied to this issue. Furthermore, in this case, we may also underspecify the 
  result of the output type, when it cannot be fully determined (mainly due to limitations 
  in the type) by leaving the output type existentially quantified, but allowing it to 
  be accompanied by the proof of some interesting property. 

\end{description} 

