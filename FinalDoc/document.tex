\documentclass[12pt, svgnames]{article} %draftclsnofoot 
%for 
%double spacing
\usepackage{amsmath}
\usepackage{amssymb}
\usepackage{fullpage}
\usepackage[round,numbers]{natbib}
\usepackage{multirow}
\usepackage{booktabs}
\usepackage{graphicx}
\usepackage{float}
\usepackage{../ltx/edcomms}
\usepackage{../ltx/setupComments}
\usepackage{hyperref}
\usepackage{geometry}
\usepackage{changepage}
\usepackage{adjustbox}
\usepackage{graphicx}
\usepackage[section]{placeins} % Prevents floats from floating across sections
\usepackage{tabularx}
\usepackage{amsfonts}
\usepackage{glossaries}
\usepackage{multirow} %% Used for Traceability matrix
\usepackage{listings}
\usepackage{calc}
\usepackage[simplified]{pgf-umlcd}
\usepackage[section]{placeins}
\usepackage{enumitem}
\usetikzlibrary{arrows.meta}
\usetikzlibrary{calc}
\usetikzlibrary{positioning}

% ---------------------------
%  Modularized Sections
%---------------------------
\usepackage{subfiles}
\usepackage[noadjust]{cite}
% \input{filename.tex}      %will format input document according to base doc, 
%%can nest
% \include{filename.tex} % cant nest
% \subfile{filename.tex} % used to load child docs
% \usepackage{standalone} %used for importing preamble of child docs
% \usepackage[final]{pdfpages.pdf} %used to insert pdf files; final 
%%format[pages=3-6]
%---------------------------
\let\Oldsubsection\subsection
\renewcommand{\subsection}{\FloatBarrier\Oldsubsection}
\let\Oldsubsubsection\subsubsection
\renewcommand{\subsubsection}{\FloatBarrier\Oldsubsubsection}

\newcounter{acnum}
\newcommand{\actheacnum}{AC\theacnum}
\newcommand{\acref}[1]{AC\ref{#1}}

\newcounter{ucnum}
\newcommand{\uctheucnum}{UC\theucnum}
\newcommand{\uref}[1]{UC\ref{#1}}

\newcounter{mnum}
\newcommand{\mthemnum}{M\themnum}
\newcommand{\mref}[1]{M\ref{#1}}
\makeglossary


% math things
\newcommand{\ra}{$\rightarrow$}



% 
\definecolor{grey}{RGB}{185,185,185}
\definecolor{applegreen}{rgb}{0.55, 0.71, 0.0}
\lstset{ %
  language=Haskell, morekeywords = {family, kind, pattern, expression},
  literate=
  {+}{{$+$}}1
  {/}{{$/$}}1 
  {*}{{$*$}}1 
  {=}{{$=\,\,\,$}}1
  {==}{{$==$}}1 
  %{/=}{{$\not\equiv$}}2
  {==}{{$\equiv$}}2
  {/=}{{$\not\equiv$}}2
  {>}{{$>$}}1 
  {<}{{$<$}}1 
  {\\}{{$\lambda$}}1
  {\\\\}{{\char`\\\char`\\}}1
  {>>}{$>>$}2 
  {:>>=}{{$:>>=$}}2
  %% {>>=}{{\hspace{6pt}\texttt{$>>=$}\hspace{6pt}}}2
  {->}{{$\rightarrow$} }2 
  {>=}{{$\geq$}}2 {<-}{{$\leftarrow$}}2
  {<=}{{$\leq$}}2 {=>}{{$\Rightarrow$} }2
  {|}{{$\mid$}}1 
  {forall}{{$\forall$}}1
  {exists}{{$\exists$}}1 
  {Nat}{{$\mathbb{N}$}}1
  {:\~:}{{$\equiv$}}2
  {\~}{{$\equiv$}}2
  {`In`}{{$\in$}}1
  {.}{{$\circ$\,\,}}1
  ,
  escapeinside={\%*}{*)},
  deletekeywords={>>,>>=,mapM,mapM_,putStrLn,putStr,toInt,show,and,sequence,String,Bool
    ,True,False,Maybe,id,Show,Ordering,Void,Just,Nothing,Int},
  morekeywords={forall, ::, :},
  postbreak={},
  breaklines=true,                
  breakatwhitespace=true,
  %postbreak={  \mbox{\textcolor{grey}{$\rightarrow$}} },
  breakindent = 0pt,
  breakautoindent = true,
  %moredelim=[is][\itshape]{"}{"},
  morestring=[b]",
  mathescape
}

\newcommand{\hstype}[2][0pt]{\attribute{\hspace*{#1}\lstinline[mathescape]|#2|}}

\newcommand{\hstypectr}[1]{\attribute{\hspace*{10pt}\lstinline[mathescape]|#1|}}

\newcommand{\hsfunc}[2][0pt]{\operation{\hspace*{#1}\lstinline[mathescape]|#2|}}


\title{\vspace*{3cm} Event control action rules for Ampersand} 
\author{Yuriy Toporovskyy (toporoy)\\ Yash Sapra (sapray) \\ Jaeden Guo 
    (guoy34)}
\date{April 17th,\ 2016}  


\begin{document}
\maketitle
\newpage


\begin{abstract}
%light description
Ampersand Tarski is a tool used to produce functional software documents based 
on business process requirements. At times, logical 
discrepancies arise when system changes occur which violate the 
restrictions set forth by the user. When a system violation occurs, one of two 
things can happen: the change that is meant to take place is adjusted so it no 
longer violates the restrictions or the changes are discarded. The purpose of 
Event condition action rules for Ampersand (EFA) was to replace the exec-engine 
that is currently used to deal with violations; unlike the exec-engine, EFA is 
automated and provides proof of correctness embedded in the code, it able to 
type SQL statements and assure no "dead-ends" occur when queries are executed.
\end{abstract}
\newpage
\tableofcontents
\newpage
\newpage
\section{Introduction}\label{Intro}

This document is a meant as a guide for EFA that includes the motivations taken 
from a business perspective, the mathematical and software foundations that 
resulted in the logical flow of EFA's design, and the testing that took place 
to assure EFA's functionality and correctness.

Currently, Ampersand is readily accessible to the public through Github and it 
is equipped with the ability to assess logical 
discrepancies on sets of data based on user-specified restrictions. Logical 
discrepancies arise when system changes occur which violate the 
restrictions set forth by the user. When a system violation occurs, one of two 
things can happen: the change that is meant to take place is adjusted so it no 
longer violates the restrictions or the changes are discarded. Ampersand is 
used to manipulate data and generate prototypes, although there is a debugger, 
certain errors still slip through. When the system rules are changed by the 
user, all data which are inconsistent with the new system must be eliminated or 
rehabilitated so it can be returned back into the system. Data inconsistencies 
are persistent bugs that can distort the product that Ampersand seeks to 
provide. 

These data inconsistencies are corrected through ECA rules which use process 
algebra (PA) to correct or discard data using violations. EFA is used to 
translate these ECA rules, execute SQL queries to correct violations and 
safeguards the database from illegal transactions.
\subsection*{Ampersand}
\subsection*{Objectives}
\subsection*{Document Guide}

\bibliographystyle{alpha}
\bibliography{references}
\end{document}

