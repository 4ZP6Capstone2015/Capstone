\documentclass[12pt]{article}
\usepackage{nopageno}
\usepackage{xcolor} % for different colour comments
\usepackage{parskip} % Space between each paragraph.
\usepackage{hardwrap} % for text length of 80 pts
\usepackage[margin=1in]{geometry}
\usepackage{../ltx/edcomms}
\usepackage{ifthen}
\usepackage{../ltx/edcomms}

%% Comments are enabled and disabled by 'draft' mode. I hacked in my own draft
%% mode (https://en.wikibooks.org/wiki/LaTeX/Macros) because the LaTeX draft
%% mode disables a bunch of things that I don't want it to. I just want it to
%% disable comments. Do not set any of this manually, just use the build script,
%% which builds both draft and final copies. Comments are enabled by default, so
%% if you build manually, you get a draft copy. 
\providecommand\draftmode{true}

\ifthenelse{\equal{\draftmode}{true}}{
\newcommand{\authornote}[3]{\textcolor{#1}{[#3 ---#2]}}
\newcommand{\todo}[1]{\textcolor{red}{[TODO: #1]}}
%\edcommstrue %% Dr. Kahl's comment package. Eventually we should migrate all
             %% comments to this.
}{
\edcommsfalse 
\newcommand{\authornote}[3]{}
\newcommand{\todo}[1]{}
}

% wss = Dr. Smith ; ds = Dr. Szymczak
\newcommand{\wss}[1]{\authornote{magenta}{SS}{#1}}
\newcommand{\ds}[1]{\authornote{blue}{DS}{#1}}



\setlength{\parindent}{15pt} % parskip sets this to 0. 15 is default.

%%%%%%%%%%%%%%%	START OF DOCUMENT %%%%%%%%%%%%%%%%%%%%

\begin{document}
\title{CS4ZP6 Problem Statement \\ Ampersand Tarski Event-Condition-Action 
Rules \\ \vspace{-2ex}} 
\author{\normalsize{Yuriy Toporovskyy,\ Yash Sapra,\ Jaeden Guo}}
\date{\vspace{-1.5em}\normalsize\today}

\thispagestyle{empty}
\maketitle\vspace{-1.5em}

Ampersand Tarski is a tool used to produce functional software documents based 
on business process requirements. It is used by system 
designers, software engineers, and those involved in business management to 
simplify the process of designing an information system that caters to the 
specific needs of their company. 
Ampersand provides users with a dynamic model that allows them to translate 
real-world problems into quantifiable requirements; the requirements can be 
tested and used to simulate modifications to a system before they are 
applied.\\ \indent
Currently, Ampersand is readily accessible to the public through Github and it 
is equipped with the ability to assess logical 
discrepancies on sets of data based on user-specified restrictions. Logical 
discrepancies arise when system changes occur which violate the 
restrictions set forth by the user. When a system violation occurs, one of two 
things can happen: the change that is meant to take place is adjusted so it no 
longer violates the restrictions or the changes are discarded. Ampersand is 
used to manipulate data and generate prototypes, although there is a debugger, 
certain errors still slip through. Some errors remain unnoticed by the system, 
even though they are logical inconsistencies that prevent the system from 
executing the changes the user implements. These inconsistencies are persistent 
bugs that can distort the product that Ampersand seeks to provide. These bugs 
occur when the user modifies or creates new system invariants, thereby causing 
conflict with existing data. The system attempts to restore equilibrium by 
modifying the data to conform to these new restrictions. Not all sets of data 
can easily be adjusted to fit new restrictions created by the user, and as a 
result there can be data that violate the restrictions but remain in the 
system, distorting the outcome of modeled solutions. Data that violate the 
restrictions are dealt with in one of two ways: elimination or 
rehabilitation.\\ \indent
The purpose of this project is to rehabilitate existing system data while 
maintaining the information system according to user specifications. The job 
of our team involves finding a creative solution that allows Ampersand to 
automatically restore system invariants, rehabilitate the data that were 
effected by the change of restrictions, and create a program that allows 
Ampersand to make the most efficient choice regarding how it wishes proceed 
based on each individual case.\\ \indent
This project is important to Ampersand stakeholders (e.g. Ampersand 
contributors and designers) because it enhances a functional requirement and 
brings Ampersand closer to becoming a finished product. For other stakeholders, 
such as Ampersand's end users, this project means a drastic decrease in the 
amount of time users would normally spend searching for and correcting system
inconsistencies.  Moreover, this project is important to our team as it 
addresses a fundamental challenge that software developers often face: the 
ability to correctly model real world problems in an artificial system that has 
no natural boundaries.
\end{document}










