\documentclass[12pt]{article}
\usepackage{nopageno}
\usepackage{xcolor} % for different colour comments
\usepackage{parskip} % Space between each paragraph.
\usepackage{hardwrap} % for text length of 80 pts
\usepackage[margin=1.2in]{geometry}
\usepackage{../ltx/edcomms}

\usepackage{ifthen}

%% Comments are enabled and disabled by 'draft' mode. I hacked in my own draft
%% mode (https://en.wikibooks.org/wiki/LaTeX/Macros) because the LaTeX draft
%% mode disables a bunch of things that I don't want it to. I just want it to
%% disable comments. Do not set any of this manually, just use the build script,
%% which builds both draft and final copies. Comments are enabled by default, so
%% if you build manually, you get a draft copy. 
\providecommand\draftmode{true}

\ifthenelse{\equal{\draftmode}{true}}{
\newcommand{\authornote}[3]{\textcolor{#1}{[#3 ---#2]}}
\newcommand{\todo}[1]{\textcolor{red}{[TODO: #1]}}
\edcommstrue %% Dr. Kahl's comment package. Eventually we should migrate all
             %% comments to this.
}{
\newcommand{\authornote}[3]{}
\newcommand{\todo}[1]{}
\edcommsfalse 
}

% wss = Dr. Smith ; ds = Dr. Szymczak
\newcommand{\wss}[1]{\authornote{magenta}{SS}{#1}}
\newcommand{\ds}[1]{\authornote{blue}{DS}{#1}}



\setlength{\parindent}{15pt} % parskip sets this to 0. 15 is default.

%%%%%%%%%%%%%%%	START OF DOCUMENT %%%%%%%%%%%%%%%%%%%%

\begin{document}
\title{CS4ZP6 Problem Statement \\ Ampersand Tarski Event-Condition-Action Rules } 
\author{Yuriy Toporovskyy,\ Yash Sapra,\ Jaeden Guo}
\date{\today}
\thispagestyle{empty}
\maketitle

Ampersand Tarski is a tool for requirement engineers, system designers and 
business users to simulate/prototype solutions to/for real world problems. 
Ampersand provides users with a dynamic mold that allows them to 
formulate percision in addressing problems through design. It allows users to 
construct an artificial world that operates based on user specified limitations.

Currently, Ampersand is live and readily accessible to the engineering 
community through github; it has the ability to access logical discrepancies on 
sets of data based on user specified limitations. Though it has the mechanism 
to manipulate data and generate prototypes, logical inconsistancies still arise 
in the data. These inconsistancies occur when the limitations imposed by the 
user changes and the data struggles to remodel itself to fit into its new 
world. Not all sets of data can easily remake itself to fit the restrictions 
imposed by the user, as a result there can be contradictions which violate one 
one or more of the restrictions. Data that violates the restrictions are dealt 
with in one of two ways: elimination or rehabilitation.

The purpose of this project focuses on the rehabilitation of data and the 
maintainance of the artificial world according to user defined limitations. 
Although Ampersand has the ability to recognize inconsistencies, it relies 
heavily on the user to manually fix inconsistencies. This project focuses on 
restoring a realistic representation of Ampersand's artificial world to be 
consistent with reality by automating the restoration of data. This is 
important to Ampersand as a whole because it automates repairs of inconsistent 
data which makes Ampersand less tedius for users and more efficient over all. 
Moreover, this project is important as it addresses a fundamental 
challenge all software engineers face, and that is how to faciliate and main an 
essence of reality in an artificial system that has no natural boundaries. 


\end{document}










