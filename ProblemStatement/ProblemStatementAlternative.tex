\documentclass[12pt]{article}
\usepackage{nopageno}
\usepackage{xcolor} % for different colour comments
\usepackage{parskip} % Space between each paragraph.
\usepackage{hardwrap} % for text length of 80 pts
\usepackage[margin=1.2in]{geometry}
\usepackage{../ltx/edcomms}

\usepackage{ifthen}

%% Comments are enabled and disabled by 'draft' mode. I hacked in my own draft
%% mode (https://en.wikibooks.org/wiki/LaTeX/Macros) because the LaTeX draft
%% mode disables a bunch of things that I don't want it to. I just want it to
%% disable comments. Do not set any of this manually, just use the build script,
%% which builds both draft and final copies. Comments are enabled by default, so
%% if you build manually, you get a draft copy. 
\providecommand\draftmode{true}

\ifthenelse{\equal{\draftmode}{true}}{
\newcommand{\authornote}[3]{\textcolor{#1}{[#3 ---#2]}}
\newcommand{\todo}[1]{\textcolor{red}{[TODO: #1]}}
\edcommstrue %% Dr. Kahl's comment package. Eventually we should migrate all
             %% comments to this.
}{
\newcommand{\authornote}[3]{}
\newcommand{\todo}[1]{}
\edcommsfalse 
}

% wss = Dr. Smith ; ds = Dr. Szymczak
\newcommand{\wss}[1]{\authornote{magenta}{SS}{#1}}
\newcommand{\ds}[1]{\authornote{blue}{DS}{#1}}



\setlength{\parindent}{15pt} % parskip sets this to 0. 15 is default.

%%%%%%%%%%%%%%%	START OF DOCUMENT %%%%%%%%%%%%%%%%%%%%

\begin{document}
\title{CS4ZP6 Problem Statement \\ Ampersand Tarski Event-Condition-Action Rules } 
\author{Yuriy Toporovskyy,\ Yash Sapra,\ Jaeden Guo}
\date{\today}
\thispagestyle{empty}
\maketitle

Ampersand Tarski is a tool for requirement engineers, system designers, and 
business users to develop prototype solutions for real world problems. 
Ampersand provides users with a dynamic model that allows them to translate 
real-world problems into quantifiable requirements, so that potential solutions 
can then be tested in order to ensure that all requirements are met.

Currently, Ampersand is live and readily accessible to the engineering 
community through Github; it has the ability to assess logical discrepancies on 
sets of data based on user specified limitations. Though it has the ability to 
manipulate data and generate prototypes, logical inconsistencies still arise in 
the system’s data. These inconsistencies occur when the user changes the 
restrictions imposed on the data and the data struggles to remodel itself to 
fit the new restrictions. Not all sets of data can easily remake themselves to 
fit the restrictions imposed by the user, and as a result there can be 
contradictions that violate one or more of the restrictions. Data that violate 
the restrictions are dealt with in one of two ways: elimination or 
rehabilitation.

The purpose of this project is to rehabilitate data and maintain the artificial 
system according to user-defined limitations. Although Ampersand has the 
ability to recognize inconsistencies, it relies entirely on the user to fix 
inconsistencies manually. The goal of this project is to automate the 
rehabilitation of data in order to restore a realistic representation of 
Ampersand's artificial system that is consistent with reality. It is important 
to automate the repair of inconsistent data because it makes Ampersand less 
tedious for users and more efficient overall.  

Moreover, this project is important as it addresses a fundamental challenge 
that all software engineers face: the need to facilitate and maintain an 
essence of reality in an artificial system that has no natural boundaries.


\end{document}










