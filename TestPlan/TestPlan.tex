\documentclass[12pt]{report}
\usepackage{xcolor} % for different colour comments
\usepackage{parskip} % Space between each paragraph.
%\usepackage{hardwrap} % for text length of 80 pts
\usepackage[margin=1.2in]{geometry}
\usepackage{hyperref}
\usepackage{../ltx/edcomms}
\usepackage{graphicx}
\usepackage[section]{placeins} % Prevents floats from floating across sections
\usepackage{natbib}%Bibtex
\usepackage{float}
\usepackage{tabularx}
\usepackage{ltablex} %% Multi page tables 
\usepackage{booktabs}
\usepackage{tabto}
\usepackage{tocloft} %% This package prevents table of contents from generating a page break
\usepackage{caption}
\usepackage{listings}
\usepackage{ifthen}
\usepackage{../ltx/edcomms}

%% Comments are enabled and disabled by 'draft' mode. I hacked in my own draft
%% mode (https://en.wikibooks.org/wiki/LaTeX/Macros) because the LaTeX draft
%% mode disables a bunch of things that I don't want it to. I just want it to
%% disable comments. Do not set any of this manually, just use the build script,
%% which builds both draft and final copies. Comments are enabled by default, so
%% if you build manually, you get a draft copy. 
\providecommand\draftmode{true}

\ifthenelse{\equal{\draftmode}{true}}{
\newcommand{\authornote}[3]{\textcolor{#1}{[#3 ---#2]}}
\newcommand{\todo}[1]{\textcolor{red}{[TODO: #1]}}
%\edcommstrue %% Dr. Kahl's comment package. Eventually we should migrate all
             %% comments to this.
}{
\edcommsfalse 
\newcommand{\authornote}[3]{}
\newcommand{\todo}[1]{}
}

% wss = Dr. Smith ; ds = Dr. Szymczak
\newcommand{\wss}[1]{\authornote{magenta}{SS}{#1}}
\newcommand{\ds}[1]{\authornote{blue}{DS}{#1}}


\usepackage{geometry}
\usepackage{changepage}
\setlength{\parindent}{15pt} % parskip sets this to 0. 15 is default.

\newcolumntype{C}[1]{>{\centering}p{#1}} %% For use with tabularx
%%%%%%%%%%%%%%%	START OF DOCUMENT %%%%%%%%%%%%%%%%%%%%
\edcommsfalse
\begin{document}

\pagenumbering{roman} %% Roman numerals before actual document starts
\begin{titlepage}\begin{center}
\thispagestyle{empty} %% No page no. on title

\vspace*{1cm}

{\Huge\textbf{Ampersand Event-Condition-Action Rules}}

\vspace{0.5cm}
{\Large Test Plan

\vspace{1.5cm}
Yuriy Toporovskyy,\ Yash Sapra,\ Jaeden Guo}
\vfill 
%% 

%% We acknowledge that this document uses material from the Volere Requirements
%% Specification Template, copyright 1995 - 2012 the Atlantic Systems Guild
%% Limited.

%% \vspace{0.8cm}
\end{center}
CS 4ZP6 \\
October 30th, 2015 \\ 
Fall 2015 / Winter 2016 
\end{titlepage}

%% Revision history

\begin{table}[ht!]\begin{center}
\caption{Revision History}  
\begin{tabular}{|l|l|l|}\hline
\textbf{Author} & \textbf{Date} & \textbf{Comment} \\\hline 
Yuriy Toporovskyy & 27 / 10 / 2015 & Reorganized document \\\hline
Yuriy Toporovskyy & 27 / 10 / 2015 & Initial version - template \\\hline
\end{tabular}
\end{center}\end{table}

\newpage

\tableofcontents
%% \listoffigures 
%% \listoftables 

\newpage
\pagenumbering{arabic} %% Arabic numerals in actual document

%%%%%%%%%%%%%%%%%%%%%%%%
%
%	1.) General Information 
%
%%%%%%%%%%%%%%%%%%%%%%%%

\chapter{General Information}\label{ch:General}

%1.1 Purpose/Summary
\section{Purpose}\label{sec:Purpose}
This document outlines the test plan for ECA for Ampersand, including our
general approach to testing, system test cases, and a specification of
methodology and constraints. This test plan specifically targets our
contribution to Ampersand, namely ECA -- elements of Ampersand, such as design
artifact generation, will not be tested. 

%% %1.3 Objectives
\section{Objectives}\label{sec:Objectives}
\subsubsection*{Preparation for testing}
The primary objective of this test plan is to collect all relevant information
in preperation of the actual testing process, in order to facilitate this process.

\subsubsection*{Communication}
This test plan intends to clearly communicate to all developers of ECA for Ampersand 
their intended role in the testing process. 

\subsubsection*{Motivation}
The testing approach is motivated by constraints and requirements outlined in the
Software Requirements Specification. This document seeks to clearly demonstrate
this motivation.

\subsubsection*{Environment}
This test plans outlines the resources, tools, and software required for the
testing process. This includes any resources needed to perform automated testing. 

\subsubsection*{Scope}
This test plan intends to better describe the scope of our contribution, ECA,
within Ampersand. 

%% %1.3  Definitions, Acronyms, and abbreviations 
\section{Acronyms, Abbreviations, and Symbols}\label{sec:Abbrev}

\begin{description}
\item[SRS] Software Requirements Specification. Document regarding requirements, constraints, and project objectives.
\item[ECA Rule] Event-Condition-Action Rule. A rule which describes how to
  handle a constraint violation in a database. See SRS for details.
\item[HUnit] A Haskell library for unit testing. See TODO: ref test tools
\item[QuickCheck] A Haskell library for running automated, randomized tests. See TODO: ref test tools
\end{description}

%%%%%%%%%%%%%%%%%%%%%%%%
%
%	2.) Plan
%
%%%%%%%%%%%%%%%%%%%%%%%%

\chapter{Plan}\label{ch:Plan}

\section{Software Description}\label{sec:SoftwareDesc}
Ampersand is a software tool which converts a formal specification of business
entities and rules, and compiles it into different design artifacts, as well as
a prototype web application.

This prototype implements the business logic in the original specification, in
the form of a relational database with a simple web-app front-end.

A particular class of relational database violations can be automatically
restored; the algorithm for computing the code to fix these violations is called
AMMBR \cite{amber}.  This class of violations is realized within Ampersand as
ECA rules -- our contribution to Ampersand will add support for ECA rules, in
both the Ampersand back-end and the generated prototype.

\edcomm{YT}{Probably should merge Test Team and Test Schedule}
\section{Test Team}\label{sec:TestTeam}

The test team which will execute the strategy outline in this document is comprised of
\begin{itemize}
\item Yuriy Toporovskyy  
\item Yash Sapra        
\item Jaeden Guo         
\end{itemize}

\section{Test Schedule}\label{sec:TestSched}

%%%%%%%%%%%%%%%%%%%%%%%%
%%%%%%%%%%%%%%%%%%%%%%%%
%%%%%%%%%%%%%%%%%%%%%%%%

\chapter{Methods and constraints}\label{ch:Methods}
\section{Methodology}\label{sec:Methodology}
\section{Test tools}\label{sec:TestTools}

\subsection{Static Typing}\label{subsec:Static}
Programming languages can be classified by many criteria, one of which is their
type systems. One such classification is static versus dynamic typing. Our
implementation language, Haskell, has a static type system. Types will be
checked at compile-time, allow us to catch errors even before the code is run,
reducing the errors that need to be found and fixed using testing techniques. 

\edcomm{YT}{``Static analysis'' is the analysis of a program without running it,
  ie, analysis of the program text. We won't be doing any static analysis, I
  believe... QuickCheck is absolutely unrelated to static analysis.}
\subsection{Formal verification}\label{subsec:FormalVer}
A part of our project deals with generating source code annotated with the proof
of derivation of that source code, which will act as a correctness proof for the
system. In particular, when we generate code to restore a database violation
using ECA rules, then the generated code will have a proof associated with it,
which details how that code was derived from the original specification given by
the user.

\subsection{Random Testing}\label{subsec:RandTest}
Random testing allows us to easily run a very large number of tests without
writing them by hand, and also has the advantage of not producing biased test
cases, like a programmer is likely to do.

We will be using QuickCheck \todo{add reference hackage} for random testing. The existing Ampersand code base
using QuickCheck for testing, therefore, using QuickCheck has the added benefit
of easier integration with the existing Ampersand code base.

QuickCheck allows the programmers to provide a specification of the program, in
the form of properties. A property is essentially a boolean valued Haskell
function of any number of arguments. QuickCheck can test that these properties
hold in a large number of randomly generated cases. QuickCheck also takes great
care to produce a large variety of test cases, and generally produces good code
coverage.

\edcomm{YT}{Don't assume that this super obvious example will make sense... better to leave it out.}

%% \begin{lstlisting}
%% quickCheck (\s -> length (take5 s) == 5)
%% Falsifiable, after 0 tests:
%% ""
%% \end{lstlisting}

%% Here we test that the length of a list after implementing ``take 5'' should be 5 however this fails when the initial list is empty or has less than 5 elements.
\subsection{Unit Testing}\label{subsec:UnitTest}
Unit testing is comprised of feeding some data to the functions being tested and
compare the actual results returned to the expected resultd.
We will be using HUnit for unit testing of the new source code \edcomm{YT}{HUnit does
not do *automated* testing.}in Ampersand. HUnit is a library providing unit
testing capabilities in Haskell.  It is an adaption of JUnit to Haskell that
allows you to easily create, name, group tests, and execute them.

\section{Requirements}\label{sec:Reqs}
\subsection{Functional requirements}\label{subsec:FunReqs}
The functional requirements for ECA for Ampersand are detailed in the SRS; they
are also briefly summarized here. Our implementation must

\begin{description}
\item[F1] provably implement the desired algorithm.
\item[F2] accept its input in the existing ADL file format.
\item[F3] produce an output compatible with the existing pipeline. 
\item[F4] be a pure function; it should not have side effects.  
\item[F5] not introduce appreciable performance degradation. 
\item[F6] provide diagnostic information about the algorithm to
the user, if the user asks for such information.
\end{description}

\subsection{Non-Functional requirements}\label{subsec:NonFunReqs}
The functional requirements for ECA for Ampersand are detailed in the SRS; they
are also briefly summarized here. Our implementation must

\begin{description}
\item[N1] produce output which will be easily understood by the typical user,
  such as a requirements engineer, and will not be misleading or confusing.  
\item[N2] be composed of easily maintainable, well documented code.
\item[N3] compile and run in the environment currently used to develop
  Ampersand.
\item[N4] annotated generated code with proofs of correctness or derivations,
  where appropriate. 
\item[N5] automatically fix database violations in the mock database of the
  prototype. 
\end{description}

\section{Data recording}\label{sec:DataRec}
\section{Constraints}\label{sec:Constraints}
\section{Evaluation}\label{sec:Evaluation}

%%%%%%%%%%%%%%%%%%%%%%%%%
%%
%%	5.) System Test Description
%%
%%%%%%%%%%%%%%%%%%%%%%%%%

\chapter{System Test Descriptions}\label{ch:SystemTests}
\newcommand{\bs}{\textunderscore}
%% Environment for system test
\newcommand{\systemTest}[8]{
\section*{#1~~~ #2}\label{sec:#1}
\addcontentsline{toc}{section}{#1~~~ #2}
\hspace{-6pt}\begin{tabular}{p{3cm}l}
\textbf{Test type}     &   #3 \\ 
\textbf{Schedule}      &   #4 \\
\textbf{Requirements}  &   #5 \\
\end{tabular}
\vspace{-12pt}\subsubsection*{Input} #6
\vspace{-12pt}\subsubsection*{Output} #7
\vspace{-12pt}\subsubsection*{Procedure} #8 
}
%% Example usage....
\systemTest{T1}  %% Test label, should be unique
{EFA Black box test}  %% Test name
{Black box/Functional, using dynamic analysis}  
{January 2016 }
{Functional}
{The input shall consist of PAClauses (i.e. a data type) that is composed of 
ECA rules. These ECA rules consist of a condition (i.e. ecaTriggr) that 
initiates a set of actions to be taken (i.e. ecaAction) based on the violation 
(i.e. ecaDelta). Please see example below for further elaboration.}
{The functional output shall be a SQL command template generated through a 
Haskell script, for each type of ECA violations}
{QuickCheck  is used to test the functionality of individual functions; it is a 
package that provides a library for testing program properties. The programmer 
is able to provide properties they want tested in their program specification, 
and QuickCheck generates numerous random cases to test that the property 
holds\cite{hackage}. }
\edcomm{JG}{So.. I originally put the entire example 
in Procedure and latex got very mad at me... so if you wanna put all that in 
here, 
feel free too or add/give a brief description (or I will) for those that do not 
want an example cause they find this self-explanatory =) 

Also, there are comments, but the comments can be taken out then shown for the 
white box test as a walk through sorta thing, am I making sense?}

\underline{Example 1.}
\paragraph{}
Input ECA Rule: No action needs to be taken when a data relation is 
deleted because 
an order placed by a client has been canceled.
	\begin{verbatim}
	ECA { ecaTriggr = On Del rel_orderedBy_Order_Client
	, ecaDelta  = vio_Delta_Order_Client
	, ecaAction = Nop []
	, ecaNum    = 2
	} 
	\end{verbatim} 	
\underline{Brief Explanation of Example 1:} \newline \newline \indent
The ecaTriggr specifies an event that takes place such as the deletion of a 
Client's order; the action to be taken is "Nop[]", which means no 
(mathematical) operations is to be performed on the data set. The ecaNum = 2, 
indicates the ECA rule that has been violated which is number 2. The deletion 
of data must be consistent and take place across all regions which it will 
affect, for example of c = (a,b) where a is an element in set a (a $\in$ A) and 
b is an element in set B (b $\in$ B), c may no longer exist. For this example, 
we shall call c, cost which is the tuple (a,b) where a is an item and b is the 
cost per weight unit of the item. Both a and b are needed to determine the cost 
because cost is based on the item and its weight per unit. If the weight no 
longer exists (i.e. it was deleted), the data table cost (c) no longer shows 
proper cost for the item in question. If the cost table was entirely devoted to 
tuples of set A and B, then the entire table needs to be adjusted for a new set 
of weights or the table must be deleted. EFA will output SQL commands to delete 
the necessary components.


Example 2 Input ECA Rule: An ECA rule that specifies that all clauses must be 
executed based on conditions. Insertion of deletion must take place to restore 
invariants. 
\begin{verbatim}
ECA {	ecaTriggr = On Ins rel_orderedAt_Order_Vendor, 
		ecaDelta  = vio_Delta_Order_Vendor, 
		ecaAction = ALL [ Do Ins rel_I_Order (EDif 
			(EIsc (ECps	(EDcD vio_Delta_Order_Vendor,EFlp(EDcD 
						vio_Delta_Order_Vendor)),
						EDcI cpt_Order),EDcI cpt_Order
							))[], Do Ins rel_I_Vendor (EDif (EIsc (ECps (EFlp 
							(EDcD 
							vio_Delta_Order_Vendor),EDcD 
							vio_Delta_Order_Vendor),EDcI cpt_Vendor),EDcI 
							cpt_Vendor))[]][], ecaNum    = 3}
\end{verbatim}

Brief Explanation of Example 2: 

\systemTest{T2}  %% Test label, should be unique
{EFA White box test} 
{White Box/Non-functional, using static analysis}  
{January 2016}
{Non-functional using static analysis}
{N/A}
{The desired output depends on the specific targeted non-functional specification}
{The test shall be performed by an third party (Dr.Kahl), where Dr. Kahl shall perform a code walk through where he will be able to confirmed the validity of the code that is written, the correctness and the maintainability of the code. Structural testing uncovers errors during implementation of the program, and focuses on how the process occurs, and evaluates structure of the program. The white box test is also implemented by teach individual of our team and shall focus on abnormal or extreme behaviour. The white box test focuses on static program analysis which can be performed without actually executing the program. The main non-functional requirement for EFA is code maintainability and correctness, which requires heavy documentation alongside the source code. }

\systemTest
{T3}{EFA System Compatibility}
{Functional }
{Dec 2015}
{F4}
{ADL File Input 1: \\
	Based on the Rule: Only members who have relevant experience may apply for 
	this 
	job \\
	Using Sets: JOBS-AVAIL, APPLICANTS, EMPLOYEES-WITH-RELEVANT-EXPERIENCE \\
	With ECA rules:\\
	\newline
	ADL Files Input 2:\\
	Based on the Rule: Only members who have relevant experience may apply for 
	this 
	job \\
	Using Sets: JOBS-AVAIL, APPLICANTS, EMPLOYEES-WITH-RELEVANT-EXPERIENCE \\
	With ECA rules: APPLICANTS must be a member of both 
	EMPLOYEES-WITH-RELEVANT-EXPERIENCE AND have a relation to (i.e. applied 
	for) 
	JOBS-AVAIL \\}
{EFA\ User\ Output\ for\ Input\ 1: \\
	Reading $<$file$>$.adl.. \\
	Generating.. \\
	Rules Done.. \\
	Sets Done.. \\
	No Errors \\
	No Violations \\
	\newline
	EFA\ User\ Output\ for\ Input\ 2: \\
	Reading $<$file$>$.adl..\\
	Generating.. \\
	Rules Done.. \\
	Sets Done.. \\
	ECA Rules Done.. \\
	No Errors \\
	No Violations \\}
{Two different version of the same script is given as input, the first is 
without ECA rules the second is with ECA rules that this project adds. Both of 
these scripts should pass through the Ampersand generator without causing 
errors or violations. If the second script which contains ECA rules 
successfully passes through each part of Ampersand then the new additions 
generated by EFA is compatible with the old Ampersand system.}





\bibliographystyle{alpha}
\bibliography{TestPlan}

\end{document}
