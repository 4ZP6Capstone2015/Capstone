\documentclass[12pt, svgnames]{article}

\usepackage{xcolor}
\usepackage{colortbl}
\usepackage{amssymb}
\usepackage{fullpage}
\usepackage[round,numbers]{natbib}
\usepackage{multirow}
\usepackage{longtable}
\usepackage{booktabs}
\usepackage{graphicx}
\usepackage{float}
\usepackage{../ltx/edcomms}
%%\usepackage{../ltx/setupComments}
\usepackage{hyperref}
\usepackage{geometry}
\usepackage{changepage}
\usepackage{adjustbox}
\usepackage{graphicx}
\usepackage[section]{placeins} % Prevents floats from floating across sections
\usepackage{tabularx}
\usepackage{amsfonts}
\usepackage{glossaries}
\usepackage{multirow} %% Used for Traceability matrix
\usepackage{listings}
\usepackage{calc}
\usepackage[simplified]{pgf-umlcd}
\usepackage[section]{placeins}
\usepackage{enumitem}

\newif\ifcomments\commentstrue
\ifcomments
\newcommand{\authornote}[3]{\textcolor{#1}{[#3 ---#2]}}
\newcommand{\todo}[1]{\textcolor{red}{[TODO: #1]}}
\else
\newcommand{\authornote}[3]{}
\newcommand{\todo}[1]{}
\fi
\newcommand{\wss}[1]{\authornote{magenta}{SS}{#1}}
\newcommand{\ds}[1]{\authornote{blue}{DS}{#1}}

\newcolumntype{L}[1]{>{\raggedright\let\newline\\\arraybackslash\hspace{0pt}}p{#1}}
%%\newcolumntype{C}[1]{>{\centering\let\newline\\\arraybackslash\hspace{0pt}}p{#1}}
%%\newcolumntype{R}[1]{>{\raggedleft\let\newline\\\arraybackslash\hspace{0pt}}p{#1}}

\begin{document}



\title{\vspace*{3cm} Test Report for ECA Rules for Ampersand} 
\author{Yuriy Toporovskyy (toporoy)\\ Yash Sapra (sapray) \\ Jaeden Guo (guoy34)}
\date{March 25th,\ 2016} 

\maketitle
\newpage
\vspace*{1cm}
\begin{table}[ht!]\begin{center}
        \caption{Revision History}  
        \begin{tabular}{|c|c|c|}\hline
            \textbf{Author} & \textbf{Date} & \textbf{Comments} \\\hline 
            Yash Sapra & 24 / 03 / 2016 & Initial draft\\\hline
	    Yash Sapra & 24 / 03 / 2016 & Performance Testing\\\hline
        \end{tabular}
    \end{center}\end{table}
\newpage

\tableofcontents

\newpage

\section{Introduction}\label{intro}

\subsection{Description}

This document details the test results of the EFA project.
This document uses the test description mentioned in the test plan.
EFA, as well as the core Ampersand system, is
currently in active development where changes occur frequently.
For this reason few tests could not performed. 
A second phase of testing will be performed 
once the EFA project is integrated into the core Ampersand. The original test plan is available in the github repository and is being actively revised in team meetings. Changes to test plan will follow soon.

\subsection{Scope}
The purpose of this document is to outline the implementation details of the 
EFA project described in the Problem Statement.
EFA is responsible for generating SQL Statements from ECA rules that will 
be used to fixed any violated invariants in the Ampersand prototype. 
The document will serve as a referral document for future software Testing and integration of EFA in the Ampersand project.

\subsection{Test Cases}
For the purpose of testing, the EFA team uses the .adl files from the ampersand-models repository. This repository contains various input files for the Ampersand Core project. Any files that compiles and runs with the core Ampersand software should also run accordingly with the EFA project.

%%%%%%%%%%%%%%%%%%%%%
%%%%%%%%%%%%%%%%%%%%%
\section{Definitions}\label{sec:Abbrev}

 \subsubsection*{Sentinel}
A test server accessible through the Ampersand website which executes a set of 
randomly generated tests on Ampersand on a daily basis.

\subsubsection*{ECA Rule}
 Event-Condition-Action Rule. A rule which describes how to handle a constraint
 violation in a database. The syntax of ECA rules is as follows:
 

\begin{lstlisting}[basicstyle=\ttfamily]
ECArule ::= 'On' ('Ins' | 'Del') 
            '(' RExpr ',' RAtom ')'
            'Do' PAclause    
\end{lstlisting}

%%%%%%%%%%%%%%%%%%%%%
%%%%%%%%%%%%%%%%%%%%% 
\section{Non-Functional Testing}

\subsection{Usability}
From a usability perspective EFA project integrates seamlessly into the current version of core Ampersand. User can use --help flag to view different options they've while generating a prototype. The `'- -print-eca-info'' flag prints the generated SQL for each ECA rule in the console. This can be useful from a development perspective in future. The Developers and Maintainers of Ampersand can use this flag to evaluate the underlying SQL accompanying each ECA rule described in the .adl file.

This test follows with the test case T11 and completed the functional requirement that the EFA project has to produce annotated code (SQL).

%%% Performance Table %%%%%
\subsection{Performance Testing}
The performance test refers to the T10 test case of the EFA project test plan. The EFA team planned to perform a degradation test to performance degradation if any. All the files were compiled with the latest version of core Ampersand and then with the EFA. The results are documented in this section. 

\begin{longtable}{|L{0.5cm}|L{5cm}|L{4cm}|L{4cm}|}
\hline
\textbf{No.} & \textbf{Input File}  & \textbf{Run-Time Without EFA project} & \textbf{Run-Time With EFA project}\\
\hline
1 & ProjectAdmin.adl	& 5.85 & 7.63\\
\hline
2 & Delivery.adl & 5.33 & 6.01\\
\hline
3 & Try1.adl  & 6.16	& 6.93\\
\hline
4 & Try2.adl & 5.95 & 6.45\\
\hline
5 & Try3.adl & 6.28 & 7.01\\
\hline
6 & Try4.adl & 6.78 & 7.44\\
\hline
7 & Try5.adl & 6.13 & 7.1\\
\hline
8 & Try6.adl & 6.16 & 7.65\\
\hline
9 & Try7.adl & 6.98 & 8.01\\
\hline
10 & Try8.adl & 7.5 & 8.65\\
\hline
11 & Try9.adl & 7.2 & 8.22\\
\hline
12 & Try10.adl & 6.33 & 7.88\\
\hline
13 & Try11.adl & 6.47 & 7.57\\
\hline
14 & Try12.adl & 7.88 & 8.68\\
\hline
15 & Try13.adl & 7.56 & 8.92\\
\hline
16 & Try14.adl & 7.11 & 8.75\\
\hline
17 & Try15.adl & 7.13 & 9.01\\
\hline
18 & Try16.adl & 6.15 & 8.01\\
\hline
19 & Try17.adl & 6.39 & 7.66\\
\hline
20 & Try18.adl & 6.04 & 7.32\\
\hline
21 & Try19.adl & 6 & 6.9\\
\hline
22 & Try20.adl & 5.62 & 6.81\\
\hline
\end{longtable}



\begin{figure}
  \centering
    \includegraphics[width=1.3\textwidth]{./Chart1}
\caption{Run Time chart for test case 1 to 10.}~\label{fig:figure1}
\end{figure}


\begin{figure}
  \centering
    \includegraphics[width=1.3\textwidth]{./Chart2}
\caption{Run Time chart for test case 11 to 22.}~\label{fig:figure2}
\end{figure}

After measuring the performance of the current version of Ampersand compared to the EFA project we found out that there is a overhead cost of generating SQL statements from the ECA rules. The average overhead time of running EFA project is 1.16 sec. 

Calculate using the formula : 
\begin{equation}
	Overhead Time(s) =\frac{\left ( \sum  Run Time with EFA - Run Time without EFA\right )}{ No. of Test Cases}
\end{equation}

Figure\ref{fig:figure1} and Figure\ref{fig:figure2} shows a comparison of running time for all the test cases. The overhead cost of integrating EFA into Ampersand will add roughly about 1 second to the time it takes to generate a prototype. However the overall running time is still under 9 seconds for all the test cases so the waiting time for the end user is still very small compared to cost and time required to create an information system otherwise.

%%%%%%%%%%%%%
\subsection{Robustness}
The language dependency of using Haskell for this project allows the Developers to pattern match against all possible inputs. The Project was tested using the `' - -Wall'' flag to turn on all the warning options in Haskell. This allowed the team to pattern match against all possible inputs, this way the project does not rely on the test cases reachable through the Ampersand test input files. 

%%%%%%%%%%%%%


%%%%%%%%%%%%%%%%%%%%%%%%%%%%%
%%%%%%%%%%%%%%%%%%%%%%%%%%%%%


\section{System Tests}
In this section we document the result of parsing ADL files through the EFA project.   


\subsection{Ampersand generates ASQL}
\begin{longtable}{|L{0.5cm}|L{2.5cm}|L{2.5cm}|L{2.25cm}|L{3cm}|L{1.75cm}|L{1.5cm}|}
\hline
\textbf{No.} & \textbf{Test Case}  & \textbf{Initial State} & \textbf{Input} & \textbf{Expected Output} & \textbf{Actual Output} & \textbf{Result}\\ 
\hline
1 & Ampersand generates ASQL & Installed EFA Ampersand &ProjectAdmin.adl & Annotated SQL & As Expected & PASS \\ 
\hline
2 & Ampersand generates ASQL & Installed EFA Ampersand &Delivery.adl & Annotated SQL & As Expected & PASS \\ 
\hline
3 &Ampersand generates ASQL & Installed EFA Ampersand &Case.adl & Annotated SQL & As Expected & PASS \\ 
\hline
\end{longtable}

\subsection{ASQL is valid}
\begin{longtable}{|L{0.5cm}|L{2.5cm}|L{2.5cm}|L{2.25cm}|L{3cm}|L{1.75cm}|L{1.5cm}|}
\hline
%\textbf{No.} & \textbf{Test Case}  & \textbf{Initial State} & \textbf{Input} & \textbf{Expected Output} & \textbf{Actual Output} & \textbf{Result}\\ 
%\hline
%1 & Ampersand generates ASQL & Installed EFA Ampersand &ProjectAdmin.adl & Annotated SQL & As Expected & PASS \\ 
%\hline
%2 & Ampersand generates ASQL & Installed EFA Ampersand &Delivery.adl & Annotated SQL & As Expected & PASS \\ 
%\hline
%3 &Ampersand generates ASQL & Installed EFA Ampersand &Case.adl & Annotated SQL & As Expected & PASS \\ 
%\hline
\end{longtable}

\subsection{EFA System Compatibility}
\begin{longtable}{|L{0.5cm}|L{2.5cm}|L{2.5cm}|L{2.25cm}|L{3cm}|L{1.75cm}|L{1.5cm}|}
\hline
\textbf{No.} & \textbf{Test Case}  & \textbf{Initial State} & \textbf{Input} & \textbf{Expected Output} & \textbf{Actual Output} & \textbf{Result}\\ 
\hline
1 & System Compatibility & Installed EFA Ampersand &ProjectAdmin.adl & No exception during generation of prototype & As Expected & PASS \\ 
\hline
2 & System Compatibility & Installed EFA Ampersand &Delivery.adl & No exception during generation of prototype & As Expected & PASS \\ 
\hline
3 & System Compatibility & Installed EFA Ampersand &Case.adl & No exception during generation of prototype & As Expected & PASS \\ 
\hline
\end{longtable}

\subsection{EFA is a pure function}
Since all functions written in
Haskell are pure, and the  Haskell type checker accepts our program hence the test is passed.

\subsection{EFA gives appropiate feedback}
This feature will be implemented on the front-end after integration into the core Ampersand project. When the prototype is run, and a violation occurs, the resulting output will look like :

\begin{verbatim}
======= Violation log entry <...>
=== ECA rule fired: <...> 
=== Delta: <...>
=== Original rule: cast;instantiates |- qualifies;comprises~
Violation occurred because rule "who's cast in roles" was not 
 satisfied. This is because "an Actor may appear in a 
 Performance of the Play only if the Actor is skilled for a 
 Role that the Play comprises"
\end{verbatim}

\subsection{EFA code walk-through}
With reference to T9 test in the test report (see page 19 of the test plan). EFA team will be doing a code walk-through with the product owners. This walk-through is not scheduled at this point. The Ampersand Team will be invited to attend the final demonstration which is to be scheduled in April.

\subsection{Sentinel Test}
After review and acceptance of the EFA project. EFA will be ran on the sentinel (see test case T13 on page 18 of Test Plan). The sentinel test is performed at regular intervals and emails developers about any failed test. This will serve as automated testing of EFA project in the future.

%%%%%%%%%%%%%%%%%%%%%%%%%%%%%
%%%%%%%%%%%%%%%%%%%%%%%%%%%%%
\section{Changes Made After testing}
After intense usability testing, the EFA team decided to format the generated SQL using a pretty printer library. The formatted SQL is indented for better readability and thereby increasing the overall usability of the EFA project.


%%\clearpage
\printglossaries
\bibliographystyle{alpha}
\bibliography{}

\end{document}

%%  LocalWords:  UML
